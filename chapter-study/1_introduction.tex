\section{Introduction}
\label{sec:3_introduction}

Performance bugs~\citep{s2e,perf.fse10,rily.perftest,perfantipattern} 
are software implementation mistakes that can cause slow execution.
The patches for performance bugs are often not too complex.
These patches can preserve the same functionality, and achieve significant performance improvement. 
Performance bugs cannot be optimized away by compiler optimization. 
Many of them escape from testing process and manifest in front of end users~\citep{xiao13:context}.
Performance bugs are common and severe. 
Mozilla developers have to fix 5 to 50 performance bugs each month in the last 10 years. 
Performance bugs have already caused several highly publicized failures, 
like the slow affordable care act system~\citep{ACA-health}.
Fighting performance bugs is solely needed.  

There are many misunderstanding of performance bugs, 
such as ``performance is taken care of by compilers and hardware'' and 
``profiling is sufficient to solve performance problems''. 
These wrong perceptions are partly the causes of today's performance-bug problem~\citep{lies}. 
The lack of understanding of performance bugs has severely limited the research and tool building in this area. 

There are many empirical studies on functional bugs~\citep{chou01empirical,characteristics.asplos08, 10yearlinux,emmett.ppopp10,
sullivan92comparison,Lu.study.fast}. 
Many of them are conducted along the following dimensions: 
root causes of functional bugs, 
how functional bugs are introduced, how to expose functional bugs, 
and how to fix functional bugs. 
These studies have provided guidance for functional-bug detection, 
functional failure diagnosis, functional-bug avoidance, and software testing. 
It is feasible and necessary to conduct similar studies on performance bugs for the following reasons: 

Firstly, it is feasible to sample performance-bug reports in the real world to conduct the empirical study. 
There are well-known open-source software with long well maintained bug databases. 
For some of them, developers explicitly mark certain bug reports in their bug databases as performance bugs. 
For example, Mozilla developers use ``perf'' tag to mark performance bugs. 
It is fairly easy for us to collect enough performance bugs in order to conduct the study.

Secondly, it is reasonable to understand whether techniques designed for functional bugs still work for performance bugs, 
before designing new techniques for performance bugs. 
In order to make this clear, 
it is necessary to follow the study dimension conducted on functional 
bugs to perform a similar empirical study for performance bugs. 

Finally, performance bugs are different from functional bugs. 
For example, performance bugs do not have failure symptoms, 
like crash or segmentation fault. 
We cannot directly leverage experience gained from combating functional bugs. 
It is necessary to understand the unique features and bug patterns for performance bugs.

This chapter makes the first, to the best of our knowledge,
comprehensive study of real-world performance bugs
based on \allbugs bugs randomly collected from the bug databases of five
representative open-source software suites (Apache, Chrome, GCC,
Mozilla, and MySQL). Our study has made the following findings.

{\bf Guidance for bug avoidance.\ }
Two thirds of the studied bugs are
introduced by developers' wrong understanding of workload
or API performance features.  
More than one quarter of the bugs arise from previously correct
code due to workload or API changes. 
To avoid performance bugs,
developers need performance-oriented
annotation systems and change-impact analysis.

{\bf Guidance for performance testing.\ } 
Almost half of the studied bugs require inputs with {\bf both} {special}
features and large scales to manifest. 
New performance-testing schemes that combine the input-generation techniques 
used by functional testing~\citep{KLEE,dart} with a consideration towards large 
scales will significantly improve the state-of-the-art.

{\bf Guidance for bug detection.\ } Recent works
~\citep{s2e,BloatFSE2008,perf.fse10,jolt,XuBloatPLDI2009,XuBloatPLDI2010}
have demonstrated the potential of performance-bug detection. 
Our study found common root causes and structural patterns of real-world
performance bugs that can help improve the coverage and accuracy 
of performance-bug detection.

{\bf Guidance for bug diagnosis.\ } 
The root-cause patterns and fix strategies for performance bugs are 
highly correlated. It is feasible to propose fix strategies automatically, based on identified root causes. 

{\bf Comparison with functional bugs.\ }
Performance bugs tend to hide for much longer time in software 
than functional bugs.
Unlike functional bugs, performance
bugs cannot all be modeled as rare events, because a 
non-negligible portion of them can be triggered by
almost all inputs.

{\bf General motivation\ }
(1) Many performance-bug patches are small.  The fact that we can
achieve significant performance
improvement through a few lines of code change motivates
researchers to pay more attention to performance bugs.
(2) A non-negligible portion of performance bugs in multi-threaded
software are related to synchronization.
Developers need tool support to avoid over-synchronization traps.
