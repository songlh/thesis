\section{Other Characteristics}
\label{sec:3_other}

{\bf Lifetime\ } We chose Mozilla to investigate the lifetime of performance bugs, 
due to its convenient CVS query interface.
We consider a bug's life to have started when its buggy code
was first written. The 36 Mozilla bugs in our study
took 966 days on average to get discovered, and another 140 days on average
to be fixed.
For comparison, we randomly sampled 36 functional bugs from Mozilla.
These bugs took 252 days on average to be discovered, which is much shorter 
than that of performance bugs in Mozilla.
These bugs took another 117
days on average to be fixed, which is a similar amount of time with those
performance bugs.


{\bf Location\ }
For each bug, we studied the location of its minimum unit of inefficiency. 
Our first finding shows that 
most performance bugs happen at call sites, and their fix are changing the usage of function calls, 
such as replacing old call sequences with new call sequences, conditionally or unconditionally skipping buggy functions or changing parameters, and so on. 
For example, {\it Retrieve Unnecessary} (Figure~\ref{fig:Apache45464}), {\it Transparent Draw} (Figure~\ref{fig:Mozilla66461}), 
{\it Intensive GC} (Figure~\ref{fig:Mozilla515287}), 
{\it Bookmark All} (Figure~\ref{fig:Mozilla490742}), and
{\it Slow Fast-Lock} (Figure~\ref{fig:MySQL38941}) are all fixed by changing function-call usage. 
This is probably because developers and compilers have already done a good job in optimizing code within each procedure. 
Therefore, future work to detect, 
diagnose and fix performance bugs should allocate more effort at call sites and procedure boundaries. 

There are also 32 bugs not fixed by changing function-call usage. 
These bugs mainly arise from two scenarios. 
In one scenario, the buggy code unit itself does not directly waste performance. 
Instead, its impact propagates to other places in the software and causes performance loss there.  
For example, the {\it No Cache} (Figure~\ref{fig:MySQL26527}) 
bug happens when MySQL mistakenly does not allocate cache. 
This operation itself does not take time, but it causes performance loss later. 
The second scenario is to optimize code units inside functions, like MySQL\#14637, 
whose patch replaces byte-wise parsing with
four-byte-wise parsing to accelerate trimming blank characters from the end of a string.

Our second finding shows that
around three quarters of bugs are
located inside either an input-dependent loop or an input-event handler. 
For example, the buggy code in Figure~\ref{fig:Mozilla515287}
is executed at every XHR completion.
The bug in Figure~\ref{fig:Mozilla66461} wastes performance
for every transparent image on a web page.
In addition, about half performance bugs involve I/Os or 
other time-consuming system calls. 
There are a few bugs whose buggy code units only execute once or twice
during each program execution. For example, the Mozilla\#110555 bug wastes
performance while processing exactly two fixed-size
default configuration files,
userChrome.css and userContent.css, during the startup of a browser.

{\bf Correlation Among Categories}
Following previous empirical studies \citep{LiASID06}, we use a statistical 
metric {\it lift} to study the correlation among characteristic categories.
The {\it lift} of category A and category B, denoted as {\it lift(AB)}, 
    is calculated 
as $\frac{P(AB)}{P(A)P(B)}$, where P(AB) is the probability of a bug belonging 
to both categories A and B. When {\it lift(AB)} equals 1,   
category A and category B are independent of each other.
When {\it lift(AB)} is greater than 1, categories A and B are 
positively correlated: when a bug belongs to A, it likely
also belongs to B. The larger the {\it lift} is, the more positively A and B
are correlated.
When {\it lift(AB)} is smaller than 1, A and B are negatively
correlated: when a bug belongs to A, it likely does not belong to B.
The smaller the {\it lift} is, the more negatively A and B are correlated.

Root cause categories are highly correlated with fix strategies. 
Among all correlations, the redundant root cause and the memorization fix strategy are 
the most positively correlated with a 3.54. 
The wrong branch selection root cause is strongly correlated with the change condition 
fix strategy with a 2.74 lift. The redundant root cause and the batch fix strategy are the third 
most positively correlated pair with a 2.36 lift. 
On the other hand, the wrong branch selection root cause has the most negative correlation 
with in-place call change, memorization and batch bug-fix strategies. 
Their lifts are all 0. 



{\bf Server Bugs vs. Client Bugs}
Our study includes 41 bugs from server applications and 69 bugs from client 
applications. To understand whether these two types of bugs have different 
characteristics, we apply chi-square test
\citep{chisquared} to each
category listed in Table~\ref{tab:3_root}, Table~\ref{tab:3_intro}, Table~\ref{tab:3_exp} and Table~\ref{tab:3_fix}.
We choose 0.01 as the significance level of our chi-square test. 
Under this setting, if we conclude that server and client bugs have different
probabilities of falling into a particular characteristic category, 
this conclusion only has a 1\% probability to be wrong. 

We find that, among all the categories listed in Table~\ref{tab:3_root}, Table~\ref{tab:3_intro}, Table~\ref{tab:3_exp} and Table~\ref{tab:3_fix},
only the synchronization issues category
is significantly different between server bugs 
and client bugs ---
synchronization issues have caused 31.7\% of server bugs and only 
4.3\% of client bugs.

