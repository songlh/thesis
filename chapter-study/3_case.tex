\section{Case Studies}
\label{sec:3_case}
We will discuss six motivating examples in this part, and 
we will answer the following questions by using these six examples. 

(1) Are performance bugs too different from traditional bugs to study along
the traditional bug-fighting process (i.e.,
bug avoidance, testing, detection, and fixing)?

(2) If they are not too different, are they too similar to be worthy of 
a study?

(3) If developers were more careful, do we still need research and tool 
support to combat performance bugs?


\begin{figure}
\codefig{Apache45464}
\caption{An Apache-HTTPD bug retrieving more than necessary data}
\label{fig:Apache45464}
\end{figure}

\begin{figure}
\codefig{Mozilla66461}
\caption{A Mozilla bug drawing transparent figures}
\label{fig:Mozilla66461}
\end{figure}

{\bf Retrieve Unnecessary (Figure~\ref{fig:Apache45464})}
Apache HTTPD developers forgot to change a parameter of API \Code{apr_stat} after an API upgrade. 
This mistake causes \Code{apr_stat} to retrieve more than necessary information 
from the file system and leads to more than ten times slowdown in Apache server. 
After changing the parameter, \Code{apr_stat} will retrieve exactly what developers originally needed
%After changing to \Code{APR_FINFO_TYPE}, \Code{apr_stat} will retrieve exactly what 
%developers originally needed through \Code{APR_FINFO_NORM}. 


{\bf Transparent Draw (Figure~\ref{fig:Mozilla66461})\ } 
Mozilla developers implemented a procedure \Code{nsImageGTK::Draw} for
figure scaling, compositing, and rendering, which is 
a waste of time for transparent figures. This problem did not catch 
developers' attention until two years later when 1 pixel by 1 pixel 
transparent GIFs became general purpose spacers widely used by Web
developers to work around certain idiosyncrasies in HTML 4.
The patch of this bug skips
\Code{nsImageGTK::Draw} when the function input is a transparent figure.

\begin{figure}
\codefig{Mozilla515287}
\caption{A Mozilla bug doing intensive GCs}
\label{fig:Mozilla515287}
\end{figure}

{\bf Intensive GC (Figure~\ref{fig:Mozilla515287})\ }
Users reported that Firefox cost 10 times more CPU than Safari on
some popular Web pages, such as gmail.com.
Lengthy profiling and code investigation revealed that
Firefox conducted an expensive
garbage collection (GC) process at the end of {\it every}
XMLHttpRequest, which is too frequent.
A developer then recalled that GC was added there
five years ago when
XHRs were infrequent, and each XHR replaced substantial portions of the 
DOM in JavaScript. However, things have changed in modern Web pages.
As a primary feature enabling Web 2.0, 
XHRs are much more common 
than they were five years ago.
%As a result, GC after every XHR became a performance bug. 
This bug is fixed by removing the call to GC.

\begin{figure}
\codefig{Mozilla490742}
\caption{A Mozilla bug with un-batched DB operations}
\label{fig:Mozilla490742}
\end{figure}

{\bf Bookmark All (Figure~\ref{fig:Mozilla490742})\ }
Users reported that Firefox hung when they clicked `bookmark all (tabs)'
with 50 open tabs.
Investigation revealed that Firefox used $N$ database transactions to bookmark
$N$ tabs, which is very time-consuming comparing with batching all bookmark
tasks into a single transaction.
Discussion among developers revealed that
there was almost no batchable database task in Firefox a few years back.  
The addition of batchable
functionalities, such as `bookmark all (tabs)' exposed this inefficiency
problem.
After replacing $N$ invocations of 
\Code{doTransact} with a single \Code{doAggregateTransact}, the hang disappears.
During patch review, developers found two more places
with similar problems and fixed them by \Code{doAggregateTransact}.

\begin{figure}
\codefig{MySQL38941}
\caption{A MySQL bug with over synchronization}
\label{fig:MySQL38941}
\end{figure}

{\bf Slow Fast-Lock (Figure~\ref{fig:MySQL38941})\ }
In order to conduct fast locking, 
MySQL synchronization-library developers implemented a \Code{fastmutex\_lock}, 
which would call library function \Code{random} to calculate spin delay. 
Unfortunately, it turns out that \Code{random} actually contains a lock, 
and this lock serializes every thread that invoke \Code{random}. 
Developers' unit test showed that invoking \Code{random} 
from multi-thread could be 40 times slower than from multi-process, 
due to the lock contention. 
This bug is fixed by replacing \Code{random} with a non-synchronized random number generator.

\begin{figure}
\codefig{MySQL26527}
\caption{A MySQL bug without using cache}
\label{fig:MySQL26527}
\end{figure}

{\bf No Cache (Figure~\ref{fig:MySQL26527})\ }
MySQL users reported that loading data into a partitioned table would be 20 times slower, 
compared with loading the same amount of data into an unpartitioned table. 
The slowness comes from the fact that cache was not used, 
and it is the branch in Figure~\ref{fig:MySQL26527} causing cache not to be allocated. 
The developer who implemented this \Code{start\_bulk\_insert} function thought that parameter 0 indicates no need of cache, 
while developer who wrote caller function thought that parameter 0 means the allocation of a large cache. 
This miscommunication causes this bug. 
The patch is to change the branch selection when using 0 as parameter. 

These six bugs can help us answer the questions asked earlier.


(1) They have similarity with traditional bugs.
For example, they are either related to usage rules of functions/APIs or related to programs' control flow, like branch,
both of which are well studied by previous work on detecting and diagnosing functional 
bugs~\citep{PRMiner05,livshits05dynamine,liblit05}.

(2) They also have interesting differences compared to traditional bugs.
For example, the code snippets in Figure 
\ref{fig:Apache45464}--\ref{fig:Mozilla490742}
turned buggy (or buggier) long after they were written,
which is rare for functional bugs.
As another example, testing designed for functional bugs
cannot effectively expose bugs like {\it Bookmark All}. 
Once the program has tried the `bookmark all (tab)' button with one or two open tabs,
bookmarking more tabs will not improve the statement or branch coverage and
will be skipped by functional testing.

(3) Developers cannot fight these bugs by themselves.
They cannot predict future workload
or code changes to avoid bugs like {\it Retrieve Unnecessary}, {\it Transparent Draw}, 
{\it Intensive GC}, and
{\it Bookmark All}. Even experts
who implemented synchronization libraries
could not avoid bugs like {\it Slow Fast-Lock}, 
given opaque APIs with unexpected performance features.
Research and tool support are needed here.

Of course, it is premature to draw any conclusion based on six bugs.
Next, we will comprehensively study \allbugs performance bugs.

