\subsection{How Performance Bugs Are Fixed}
\label{sec:char_fix}

\begin{table*}[tb!]
\begin{adjustwidth}{-.5in}{-.5in}
\scriptsize
\centering
{

\begin{tabular}{|l|c|c|c|c|c|c|}
\toprule
\multicolumn{1}{|c|}{\bf Granularity for Performance Bugs} &Apache&Chrome&GCC&Mozilla&MySQL&Total\\
\midrule
\multicolumn{1}{|l|}{Performance bugs happening at function boundaries}
&22&8&4&29&18&81\\
\midrule
\multicolumn{1}{|l|}{Performance bugs not happening at function boundaries}
&3&2&7&7&10&29\\
\bottomrule

\end{tabular}
}
\end{adjustwidth}
\caption{Granularity categorization for performance bugs in Section~\ref{sec:char_fix}}
\label{tab:location}
\end{table*}

We find that most performance bugs happen at call sites, and they are fixed by changing the usage of function calls, 
such as replacing old call sequences with new call sequences, conditionally or unconditionally skipping buggy functions or changing parameters, and so on. 
For example, {\it Transparent Draw} (Figure~\ref{fig:moz66461}), {\it Intensive GC} (Figure~\ref{fig:moz515287}), 
{\it Bookmark All} (Figure~\ref{fig:uncoord2}), and
{\it Slow Fast-Lock} (Figure~\ref{fig:mysql38941}) are all fixed by changing function call usage.
This is probably because developers and compilers have already done a good job in optimizing code within each procedure. 
Therefore, future work to detect, diagnose and fix performance bugs should allocate more effort at call sites and procedure boundaries. 

There are also 29 bugs not fixed by changing function call usage. 
These bugs mainly arise from two scenarios. 
In one scenario, the buggy code unit itself does not directly waste performance. 
Instead, its impact propagates to other places in the software and causes performance loss there. 
For example, the {\it No Cache} (Figure~\ref{fig:mysql26527}) 
bug happens when MySQL mistakenly does not allocate cache. 
This operation itself does not take time, but it causes performance loss later. 
The second scenario is to optimize code units inside functions, like MySQL14637, whose patch replaces byte-wise parsing with
four-byte-wise parsing to accelerate trimming blank characters from the end of a string.


\begin{table*}[tb!]
\begin{adjustwidth}{-.5in}{-.5in}
\scriptsize
\centering
{
\begin{tabular}{|l|c|c|c|c|c|c|}
\toprule
\multicolumn{1}{|c|}{\bf How to Fix Performance Bugs} &Apache&Chrome&GCC&Mozilla&MySQL&Total\\
\midrule
\multicolumn{1}{|l|}{Change Condition: {a condition is added or modified} }
&1&4&7&14&10&36\\
\midrule
\multicolumn{1}{|l|}{In-place Call Change: {replace call sequences in the exact same place} }
&9&1&1&8&8&27\\
\midrule
\multicolumn{1}{|l|}{Memorization: {reuse results from previous computation}}
&9&0&3&3&3&18\\
\midrule
 \multicolumn{1}{|l|}{Batch: {batch computation to eliminate  redundancy}}
&4&3&0&6&1&14\\
\midrule
\multicolumn{1}{|l|}{Others: all the bugs not belonging to the above four categories}
&2&2&1&5&6&16\\
\bottomrule
\end{tabular}
}
\end{adjustwidth}
\caption{How to fix performance bugs in Section~\ref{sec:char_fix}}
\label{tab:fix}
\end{table*}


There are four common strategies in fixing performance bugs, as shown in Table~\ref{tab:fix}.

\underline{\it Change Condition} The most common fix strategy is {\it Change Condition}. 
It is used in 36 patches, in which code units that do not always generate useful 
results are conditionally skipped, 
a fast path is changed to be executed, or the same functionality is realized in a more efficient way. 
For example, Draw is conditionally skipped to fix {\it Transparent Draw} (Figure~\ref{fig:moz66461}), 
and cache will be used to fix {\it No Cache} (Figure~\ref{fig:mysql26527} ).

\underline{\it In-place Call Change} The second most common strategy is {\it In-place Call Change}. 
In this strategy, developers replace or reorganize the call sequence in the exact same place. 
The performance gain can be achieved whenever changed codes are executed. 
For example, in order to fix Mozilla103330, 
developers replace the setLength and Append call combination with Assign.

\underline{\it Memorization} fixes 18 bugs by reusing results 
from previous calculations. 
For example, in order to fix Mozilla409961, 
developers move do\_QueryInterface outside the buggy loop, instead of calling it in each iteration.  

\underline{\it Batch} strategy is used in 14 patches. 
For example, {\it Bookmark All} in Figure~\ref{fig:uncoord2} is fixed by this strategy.   

\underline{\it Are patches complicated?}
Most performance bugs in our study can be fixed through simple changes. 
In fact, 42 out of 110 bug patches contain five or fewer lines of code changes. 
The median patch size for all examined bugs is 8 lines of code.
The small patch size is a result of the above fixing strategies.
Many change condition patches are small. Most in-place call change, memorization, and batch patches do not require implementing new function calls.

