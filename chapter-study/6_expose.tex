\section{How Performance Bugs Are Exposed}
\label{sec:3_exp}

\begin{table*}[tb!]
\begin{adjustwidth}{-1in}{-1in}
\scriptsize
\centering
{
\begin{tabular}{lcccccc}
\toprule
\multicolumn{1}{c}{\bf How Performance Bugs Are Exposed} &Apache&Chrome&GCC&Mozilla&MySQL&Total\\
\midrule
\multicolumn{1}{l}{{\bf Always Active:} almost every input on every platform can trigger this bug}
&2&3&0&6&5&16\\
\midrule
\multicolumn{1}{l}{{\bf Special Feature:} need special-value inputs to cover specific code regions}
&18&7&11&23&17&76\\
\midrule
\multicolumn{1}{l}{{\bf Special Scale:} need large-scale inputs to execute a code region many times}
&18&2&10&21&18&69\\
\midrule
\multicolumn{1}{l}{{\bf Feature+Scale:} the intersection of Special Feature and Special Scale}
&13&2&10&14&12&51\\
\bottomrule
\end{tabular}
}
\end{adjustwidth}
\caption{How performance bugs are exposed in Section~\ref{sec:3_exp}.}
\label{tab:3_exp}
\end{table*}


We define exposing a performance bug as causing a perceivably negative performance impact, 
following the convention used in most bug reports. Our study demonstrates several unique challenges for performance testing.

\underline{\it Always Active Bugs\ } 
A non-negligible portion of performance bugs are almost always active.
They are located at the start-up phase, the shutdown phase, or other places 
that are exercised by almost all inputs. 
They could be very harmful in the long term, because
they waste performance at every deployment site during every run of a program.
Many of these bugs were caught during comparison with other software
(e.g., Chrome vs. Mozilla vs. Safari).

Judging whether performance bugs have manifested is a unique challenge in 
performance testing.

\underline{\it Input Feature \& Scale Conditions\ } 
About two thirds of performance bugs need inputs with special features
to manifest. Otherwise, the buggy code units cannot be touched.
Unfortunately, this is not what black-box testing is good at.
Much manual effort will be needed to design test inputs, a problem
well studied by past research in functional testing \citep{KLEE,s2e}.


About two thirds of performance bugs need large-scale inputs to manifest in
a perceivable way. These bugs cannot be effectively exposed if 
software testing executes each buggy code unit only once,
which unfortunately is the goal of most functional testing.

Almost half of the bugs need inputs that have special features {\bf and} 
large scales to manifest. 
For example, to trigger the bug shown in Figure~\ref{fig:Mozilla515287}, the user
has to click `bookmark all' button (i.e., special feature) with many open
tabs (i.e., large scale).

