\section{How Performance Bugs Are Introduced}
\label{sec:3_introduce}

\begin{table*}[tb!]
\begin{adjustwidth}{-1in}{-1in}
\scriptsize
\centering
{
\begin{tabular}{lcccccc}
\toprule
\multicolumn{1}{c}{\bf How Performance Bugs Are Introduced} &Apache&Chrome&GCC&Mozilla&MySQL&Total\\
\midrule
\multicolumn{1}{l}{{\bf Workload Issues:} {developers' workload assumption is wrong or out-dated}}
&15&4&7&21&10&57\\
\midrule
\multicolumn{1}{l}{{\bf API Issues:} {misunderstand performance features of functions/APIs}}
&6&2&1&10&9&28\\
\midrule
\multicolumn{1}{l}{{\bf Others:} all the bugs not belonging to the above two categories}
&4&4&3&6&9&26\\
\bottomrule

\end{tabular}
}
\end{adjustwidth}
\caption{How performance bugs are introduced in Sections~\ref{sec:3_introduce}.}
\label{tab:3_intro}
\end{table*}

We have studied the discussion among developers in bug databases and checked the 
source code of different software versions to understand how bugs are introduced. 
Our study has particularly focused on the challenges faced by developers in writing 
efficient software, and features of modern software that affect the introduction
of performance bugs. 

Our study shows that developers are in a great need of tools that can help them 
avoid the following mistakes.

\underline{\it Workload Mismatch\ }
Performance bugs are most frequently introduced when 
developers' workload understanding does not match with the reality.

Our further investigation shows that the following challenges
are responsible for most workload mismatches.

Firstly, the input paradigm could shift {\it after}
code implementation. For example, the HTML standard change and new trends in 
web-page content led to {\it Transparent Draw} and
{\it Intensive GC}, shown
in Figure~\ref{fig:Mozilla66461} and Figure~\ref{fig:Mozilla515287}.

Secondly, software workload has become much more diverse and complex than 
before.
A single program, such as Mozilla, may face various types of workload issues:
the popularity of transparent figures on web pages led
to {\it Transparent Draw} 
in Figure \ref{fig:Mozilla66461}; the high frequency of XMLHttpRequest led to
{\it Intensive GC}
in Figure \ref{fig:Mozilla515287}; users' habit of not changing the default
configuration setting led to Mozilla\#110555. 

The increasingly dynamic and diverse workload of modern software will lead
to more performance bugs in the future.

\underline{\it API Misunderstanding\ }
The second most common reason is that
developers misunderstand the performance feature of certain
functions. This occurs for 28 bugs in our study.

Sometimes, the performance of a function is sensitive to the value of a
particular parameter, and developers happen to use performance-hurting values. 

Sometimes, developers use a function to perform task $i$, and are unaware of
an irrelevant task $j$ 
conducted by this function that hurts performance but not functionality. 
For example, MySQL developers did not know the synchronization inside
\Code{random} and introduced the {\it Slow Fast-Lock} bug shown in 
Figure~\ref{fig:MySQL38941}. 

Code encapsulation in modern software leads to many APIs with poorly 
documented performance features. 
We have seen developers explicitly complain about this issue 
\citep{apache45396}. It will lead to more performance bugs in the future.

\underline{\it When a bug was not buggy\ }
An interesting trend is that 29 out of \allbugs bugs were not born buggy.
They became inefficient long after they were written due to workload shift,
such as that in {\it Transparent Draw} and {\it Intensive GC}
(Figures~\ref{fig:Mozilla66461} and \ref{fig:Mozilla515287}), 
and code changes in other part of the software, such as
that in Figure~\ref{fig:Apache45464}.
In Chrome\#70153, when GPU accelerator became available, software
rendering code became redundant.
Many of these bugs went through regression testing without being caught.
