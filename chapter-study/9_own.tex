\section{Guidance for My Thesis Work}
\label{sec:3_own}

{\bf Performance-bug Detection}
Our study provides several motivations for our own rule-based bug detection work, 
which will be discussed in detail in Chapter~\ref{chap:detec}. 
Most performance bugs loss performance at function-call sites (Section~\ref{sec:3_other}), 
more than one fourth of performance bugs are introduced by misunderstanding API (Section~\ref{sec:3_introduce}), 
and more than one fourth of performance bugs are fixed by in-place call changes (Section~\ref{sec:3_fix}). 
We could detect similar inefficient call usage to find new bugs, and propose more efficient call sequences with the same functionalities.
Because some performance bugs are always active (Section~\ref{sec:3_exp}), 
performance bugs cannot be modeled as rare events.
Automatically inferring efficiency rules~\citep{engler01bugs} may not be feasible for performance bugs. 
Patches for performance bugs are simple (Section~\ref{sec:3_fix}), and they follow limited fix strategies. 
It is feasible to extract efficiency rules from these patches. 

{\bf Performance Failure Diagnosis}
Our study also provides guidance for our performance failure diagnosis work (Chapter~\ref{chap:sd} and Chapter~\ref{chap:ldoctor}). 
A non-negligible portion of performance bugs are caused by wrong branch selection (Section~\ref{sec:3_introduce}). 
Statistical debugging, leveraging branch predicate, can well diagnose functional bugs with similar root causes. 
It is promising to explore how to apply statistical debugging to diagnose performance bugs. 
The three common root causes are wrong branch, resultless and redundancy (Section~\ref{sec:3_root}). 
Our diagnosis projects should focus on bugs caused by these three root causes. 
Root causes and fix strategies are highly correlated (Section~\ref{sec:3_other}).
It is feasible to automatically provide fix suggestions based on identified root causes. 

