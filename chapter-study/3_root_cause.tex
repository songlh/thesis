\subsection{Root Causes of Performance Bugs}
\label{sec:taxonomy_howwaste}

There are a large variety of potential root causes for inefficient code, 
such as poorly designed algorithms, non-optimal data structures, cache-unfriendly data layouts, etc. 
Our goal here is not to discover previously unheard-of root causes, 
but to check whether there are common root-cause patterns among 
real-world performance bugs that bug detection and diagnosis work can focus on.

Our study shows that the majority of real-world performance bugs in our study are covered by only a couple of root-cause categories (Table~\ref{tab:root_cause}).

\underline{\it Wrong Branch Selection} 
A non-negligible portion of performance bugs are branch-related. 
There are three scenarios in which bugs in this category are triggered. 
In the first case, some slow code paths would be executed. 
For example, when Mozilla231300 is triggered, 
Firefox would use separate system calls to move files 
in the same directory one by one instead of using one system call to move them altogether. 
In the second case, some unnecessary work will be conducted. 
For example, when Mozilla258793 is triggered, 
Firefox will call draw functions for background figures, which do not exist. 
In the last case, some desired functionalities will be realized in an inefficient way, 
such as the {\it No Cache} example shown in Figure~\ref{fig:mysql26527}. 

\underline{\it Workless} 
Around one third of performance bugs are caused by workless codes. 
Buggy codes rarely generate results whenever executed, 
or they rarely generate results under bug-triggering inputs. 
Bugs in this category can be divided along two dimensions: 
according to different workless granularities, 
workless bugs can be divided into loop-related bugs and not loop-related bugs; 
based on whether semantic information is needed to judge whether codes are workless, 
bugs in this category can be divided into semantic workless, and non-semantic workless.
For example, inside the GC function, there is a loop to scan all heap objects 
and free objects with reference number 0. 
For the {\it Intensive GC} example in Figure~\ref{fig:moz515287}, this loop rarely deallocates objects. 
{\it Intensive GC} example  is loop-related and non-semantic workless.
{\it Transparent Draw} in Figure~\ref{fig:moz66461} is not loop-related, 
and we need semantic information to know that drawing a transparent figure is workless.

\underline{\it Redundancy} 
Redundancy means repeated work. 
Intuitively, we can remove repeated work and improve performance. 
According to different code granularity to observe redundant work, 
bugs in this category could be divided into redundant snippets, 
cross-iteration redundancy, and cross-loop redundancy.  
For example, in Chrome70153, 
both software and GPU will render the same video redundantly, 
and this bug is categorized as redundant snippets. 
In the {\it Bookmark All} example shown in Figure~\ref{fig:uncoord2}, 
Firefox will start, commit, and destroy a transaction for each tab in each iteration, 
and there is a lot of redundant work across different iterations. 
There are also bugs, for which redundant work is not inside one loop
but across different execution of the same loop, like Mozilla35294. 


\underline{\it Synchronization Issues}
Unnecessary synchronization that intensifies thread competition 
is also a common root cause, as shown in the {\it Slow Fast-Lock} bug (Figure~\ref{fig:mysql26527}). 
These bugs are especially common in server applications, contributing to 3 out of 15 Apache server bugs and 6 out of 26 MySQL server bugs.


\underline{\it Others}
There are also bugs that do not fall into above categories. 
For those bugs, developers find more efficient methods to optimize original codes. 
For example, in order to accelerate the startup of GPU process, Chrome developers use one extra thread to collect expensive GPU information. 
For MySQL14637, MySQL developers replace byte-wise parsing with four-byte-wise parsing to accelerate trimming blank characters from the end of a string. 

\begin{table*}[tb!]
\begin{adjustwidth}{-.5in}{-.5in}
\scriptsize
\centering
{

\begin{tabular}{lcccccc}
\toprule
\multicolumn{1}{c}{\bf Root Causes for Performance Bugs} &Apache&Chrome&GCC&Mozilla&MySQL&Total\\
\midrule
\multicolumn{1}{l}{Wrong Branch Selection: wrong branch selection to cause performance loss}
&2&0&1&6&7&16\\
\midrule
\multicolumn{1}{l}{Workless: not generate desired results}
&4&4&3&18&9&38\\
\midrule
\multicolumn{1}{l}{Redundancy: repeated work causing performance loss}
&13&3&3&6&2&27\\
\midrule
\multicolumn{1}{l}{Synchronization Issues: inefficient synchronization among threads}
&5&0&0&1&6&12\\
\midrule
\multicolumn{1}{l}{Others: all the bugs not belonging to the above four categories}
&4&3&4&6&7&24\\
\bottomrule

\end{tabular}
}
\end{adjustwidth}
\caption{Root cause categorization in Section~\ref{sec:taxonomy_howwaste}}
\label{tab:root_cause}
\end{table*}








%\underline{\it Control Flow of Performance Bugs}
%As we discussed earlier, there are performance bugs related to branch selection. 
%There are also bugs related loops, including bugs caused by workless loops, cross-iteration redundant bugs, cross-loop redundant bugs, 
%and some bugs categorized as Others. 
%Detailed numbers of control flow categorization for performance bugs 
%and user perceived performance bugs are shown in Table~\ref{tab:cfg} and Table~\ref{tab:cfg_user} respectively. 
%There are more than 80\% of user perceived performance bugs either related to branches or related to loops.
%Future performance bug diagnosis tools can focus on these two types of bugs.   



