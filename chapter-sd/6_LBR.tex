\section{Production-run statistical debugging}
\label{sec:lbr}
In-house performance diagnosis discussed in Section \ref{sec:inhouse} assumes
that users file a detailed bug report and developers can repeat the 
performance problem at the development site. Unfortunately, this does not always
happen. In many cases, production-run users only send 
back a simple automatically generated report, claiming that a failure has 
happened, together with a small amount
of automatically collected run-time information. 

The key challenge of diagnosing production-run failures is how to satisfy the
following three requirements simultaneously:

\begin{enumerate}
\item Low run-time overhead. 
The diagnosis tool will not be accepted by end users, if it incurs too much
slow down to each production run.

\item High diagnosis capability. 
The diagnosis tool is useful to developers only when it can accurately 
provide failure root cause information.

\item Short diagnosis latency. 
Short diagnosis latency can speed up patch design and improve system
availability.
\end{enumerate}

This section discusses this issue
in the context of performance bugs.

\subsection{Design}
The state of the art on production-run
functional bug diagnosis \citep{liblit03,liblit05,CCI,joy.asplos13} proposes
to satisfy the first two requirements (i.e., low overhead and high capability)
by combining  
\textit{sampling} techniques with
statistical debugging.
By randomly sampling predicates at run time, the overhead can be lowered;
by processing predicates collected from many failure and success runs
together, the diagnosis capability can be maintained for the diagnosis of most
functional bugs \citep{liblit03,liblit05,CCI,joy.asplos13}.
The only limitation is that sampling could affect diagnosis latency 
--- the same failure needs to occur for many times until sufficient information
can be sampled. This is especially a problem for software that is not widely
deployed and bugs that do not manifest frequently.
We plan to follow this approach and apply it for production-run performance
diagnosis.

Different from production-run functional failure diagnosis
\citep{liblit03,liblit05,CCI,joy.asplos13}, production-run performance diagnosis
needs to have a slightly different failure-reporting process.
Traditional functional failure diagnosis assumes that a profile of
sampled predicates will be collected after
every run. This profile will be marked as
\textit{failure} when software encounters typical failure symptoms such as 
crashes, error messages, and so on; the profile will be marked
as \textit{success} otherwise. The same process does not apply to performance
failures, because most performance failures are observed through
comparisons across runs, as discussed in Section \ref{sec:study}. 

To adapt to the unique way that performance problems are observed, we 
expect that users will explicitly mark a profile as \textit{success}, 
\textit{failure}, or \textit{do-not-care} (the default marking), when they 
participate in production-run performance diagnosis. For most performance 
problems (i.e., those problems observed through comparisons), do-not-care
profiles will be ignored during statistical debugging. For performance problems
that have non-comparison-based symptoms (i.e. application freeze), all
profiles collected from production runs will be considered during statistical
debugging.

One issue not considered in this paper is \textit{failure bucketing}. That is, 
how to separate failure (or success) profiles related to different software 
defects. This problem is already handled by some statistical models 
\citep{liblit03,liblit05} that can discover multiple failure predictors
corresponding to different root causes mixed in one
profile pool, as well as some failure bucketing techniques
\citep{hunt.sosp09}
that can roughly cluster profiles based on likely root causes.
Of course, performance diagnosis may bring new challenges to these
existing techniques. We leave this for future research.

\subsection{Experimental evaluation}

Our evaluation will aim to answer two key questions:

\begin{enumerate}
\item Can sampling lower the overhead and maintain the capability of
performance-related statistical debugging? A positive answer would indicate
a promising approach to production-run performance diagnosis.

\item What is the impact of sampling to diagnosis latency?
Traditionally, if we use 1 out of 100 sampling rate, we need hundreds of failure
runs to achieve good diagnosis results. Since many performance bugs lead
to repeated occurrences of an event at run time, it is possible that fewer
failure runs would be sufficient for performance diagnosis. If this heuristic
is confirmed, we will have much shorter diagnosis latency than traditional
sampling-based failure diagnosis for functional bugs.
\end{enumerate}


\subsubsection{Methodology}
\paragraph{Benchmarks and inputs}
We reuse the same set of benchmarks shown in Table~\ref{tab:app_bug}. 
We also use the same methodology to generate inputs and drive success/failure
runs. The only difference is that for the four performance problems that users
do not report any good inputs, we will use completely random inputs to produce
success-run profiles.

\paragraph{Tool implementation}
To sample return predicates, we directly use CBI \citep{liblit03,liblit05}.
CBI instruments program source code to conduct sampling.
Specifically, CBI instrumentation keeps a global countdown to decide how many 
predicates can be skipped before next sample. 
When a predicate is sampled, 
the global countdown is reset to a new value based on a geometric distribution 
whose mean value is the inverse of the sampling rate. 

To sample branch predicates, we directly use CBI for benchmarks written in 
C. For all MySQL and some Mozilla benchmarks that are written in C++, since
CBI does not work for C++ code, we conduct sampling through
hardware performance counters following the
methodology described in previous work \citep{joy.asplos13}.
Specifically, hardware performance
counters are configured so that an interrupt will be triggered every $N$
occurrences of a particular performance event (e.g, branch-taken event), with no changes to the program.

\paragraph{Metrics and settings}
We will evaluate all three key metrics for failure diagnosis:
(1) run-time overhead, measured by the slow down caused by information
collection at every run;
(2) diagnosis capability,
measured by whether top ranked failure predictors are 
related to failure root causes, as discussed in 
Section~\ref{sec:inhouse_results}.
(3) diagnosis latency,
measured by how many failure runs are needed 
to complete the diagnosis. 

By default, we keep the sampling rate at roughly
1 out of 10000
and use samples
collected from 1000 failure runs and 1000 success runs for failure diagnosis. 

In addition to experiments under the default setting, we also evaluate the 
impact of different numbers of failure/success runs, ranging from 10 to 1000,
while keeping the sampling rate fixed, and evaluate the impact of different
sampling rates, ranging from roughly 1 out of 100 to roughly 1 out of 100000,
while keeping the number of failure/success runs fixed.
Particularly, we will try using only 10 success runs
and 10 failure runs, under the default sampling rate, to see if we can achieve 
good diagnosis capability,
low diagnosis latency, and low run-time overhead \emph{simultaneously}. 

Since sampling is random, we have repeated our evaluation for several rounds to confirm that all
the presented results are stable.

For every performance problem benchmark, the results presented below are 
obtained under the
combination of predicate and statistical model that is shown to be (most) 
effective
in Table \ref{tab:in-house} (Section \ref{sec:inhouse}).
That is, basic model plus branch predicates are used for seven benchmarks;
basic model plus return predicates are used for one benchmark;
$\Delta$LDA model plus branch$_{\text{loop}}$ predicates are used for the remaining twelve  
benchmarks, including GCC12322.
Since sampling can only lower overhead and cannot
improve the diagnosis capability, those combinations that fail to deliver
useful diagnosis results in Table \ref{tab:in-house} still fail to deliver
useful diagnosis results in our sampling-based evaluation.

\subsubsection{Results}

\begin{table}[h!]
  \centering
  \scriptsize
  \newcommand{\Yes}[1]{\checkmark{}$_#1$}
  \newcommand{\No}[0]{-}
  \begin{tabular}{lccccc}
    \toprule     
   {\bf BugID}             &  \multicolumn{4}{c}{ Diagnosis Capability}   & Overhead \\
                           
    \cmidrule(lr){2-5}
    (\# of runs)                 &  (10)     &   (100)    &    (500)    & (1000)     &   per run\\
    \midrule 

    Mozilla258793                & \No       & \Yes{1}    &  \Yes{1}    & \Yes{1}    &   2.39\%        \\
    Mozilla299742                & \No       & \No        &  \Yes{1}    & \Yes{1}    &   4.27\%        \\
    Mozilla347306                & \Yes{1}   & \Yes{1}    &  \Yes{1}    & \Yes{1}    &   1.42\%  \\
    Mozilla416628                & \Yes{1}   & \Yes{1}    &  \Yes{1}    & \Yes{1}    &   2.03\%   \\
    \midrule
    MySQL15811                   & \Yes{1}   & \Yes{1}    & \Yes{1}     & \Yes{1}    &  2.25\% \\
    MySQL26527                   & \No       & \No        & \Yes{1}     & \Yes{1}    &  6.05\% \\
    MySQL27287                   & \Yes{1}   & \Yes{1}    & \Yes{1}     & \Yes{1}    &  3.02\% \\
    MySQL40337                   & \No       & \Yes{1}    & \Yes{1}     & \Yes{1}    &  2.69\% \\
    MySQL42649                   & \No       & \No        & \Yes{2}     & \Yes{1}    &  6.10\% \\
    MySQL44723                   & \No       & \Yes{1}    & \Yes{1}     & \Yes{1}    &  3.16\% \\
    \midrule
    Apache3278                   & \No       & \No        &  \No        & \No        &  0.23\%         \\
    Apache34464                  & \Yes{3}   & \Yes{3}    & \Yes{3}     & \Yes{3}    &  0.18\%     \\
    Apache47223                  & \Yes{1}   & \Yes{1}    & \Yes{1}     & \Yes{1}    &  0.13\%         \\
    Apache32546                  & \Yes{1}   & \Yes{1}    & \Yes{1}     & \Yes{1}    &  0.38\%     \\
    \midrule
    GCC1687                      & \Yes{1}   & \Yes{1}    &  \Yes{1}    & \Yes{1}    &  0.80\%  \\
    GCC8805                      & \Yes{4}   & \Yes{4}    &  \Yes{4}    & \Yes{4}    &  1.81\%  \\
    GCC15209                     & \No       & \No        &  \Yes{1}    & \Yes{1}    &  2.37\%        \\
    GCC21430                     & \Yes{1}   & \Yes{1}    &  \Yes{1}    & \Yes{1}    &  7.55\%  \\
    GCC46401                     & \Yes{2}   & \Yes{2}    &  \Yes{2}    & \Yes{2}    &  2.91\%  \\
    GCC12322                     & \No       & \No        &  \No        & \No        &  2.33\%  \\

    \bottomrule
   \end{tabular}
  %\nocaptionrule
  \caption{Run-time overhead and diagnosis capability evaluated with the default sampling rate (1 out of 10000); 10, 100, 500, 1000 represents the different numbers of success/failure runs used for diagnosis.}
  \label{tab:LBR}
\end{table}



\begin{table*}
  \centering
  \scriptsize
  \newcommand{\Yes}[1]{\checkmark{}$_#1$}
  \newcommand{\No}[0]{-}
  \begin{adjustwidth}{-.9in}{-.9in}
  {
  \begin{tabular}{lcccccccccccc}
    \toprule     
   {\bf BugID} & \multicolumn{4}{c}{ Diagnosis Capability} &\multicolumn{4}{c}{Overhead} & \multicolumn{4}{c}{Avg. \# of sampled predicates} \\
                           
    \cmidrule(lr){2-5}
    \cmidrule(lr){6-9}
    \cmidrule(lr){10-13}
    (sampling rate)  &($\frac{1}{10^2}$)&($\frac{1}{10^3}$)&($\frac{1}{10^4}$)& ($\frac{1}{10^5}$)  &($\frac{1}{10^2}$) &($\frac{1}{10^3}$)&($\frac{1}{10^4}$)  & ($\frac{1}{10^5}$)  & ($\frac{1}{10^2}$)  & ($\frac{1}{10^3}$)    & ($\frac{1}{10^4}$)   &  ($\frac{1}{10^5}$)         \\
    \midrule 

    Mozilla258793    &  *    & \Yes{1}  & \Yes{1}   & \Yes{1}     & *       & 24.36\% &  2.39\%     &  1.84\%     & *         & $1.42*10^6$   & $1.45*10^5$   &  $1.49*10^4$           \\
    Mozilla299742    &  *    & \Yes{1}  & \Yes{1}   & \Yes{2}     & *       & 30.84\% &  4.27\%     &  4.16\%     & *         & $1.87*10^5$   & $1.77*10^4$   &  $1.82*10^3$\\
    Mozilla347306    & \Yes{1} & \Yes{1}  & \Yes{1}   & \Yes{1}     & 69.73\%   & 8.27\%  &  1.42\%     &  0.56\%     &$7.13*10^6$  & $7.13*10^5$   & $7.13*10^4$   &  $7.13*10^3$\\
    Mozilla411722    & \Yes{1} & \Yes{1}  & \Yes{1}   & \Yes{1}     & 24.64\%   & 4.31\%  &  2.03\%     &  1.36\%     &$8.18*10^5$  & $8.18*10^4$   & $8.17*10^3$   &  816.56  \\
    \midrule
    MySQL15811       & *     & \Yes{1}  & \Yes{1}   & \Yes{1}     & *       & 7.65\%  &  2.25\%     &  1.53\%     & *         &$3.67*10^5$    &  $1.67*10^5$  & $1.66*10^4$          \\
    MySQL26527       & *     & \Yes{1}  & \Yes{1}   & \No         & *       & 6.40\%  &  6.05\%     &  4.53\%     & *         &$3.23*10^3$    &  921.41       & 92.60 \\
    MySQL27287       & *     & \Yes{1}  & \Yes{1}   & \Yes{1}     & *       & 4.63\%  &  3.02\%     &  0.61\%     & *         &$2.52*10^6$    &  $1.15*10^6$  & $1.19*10^5$\\
    MySQL40337       & *     & \Yes{1}  & \Yes{1}   & \No         & *       & 10.88\% &  2.69\%     &  2.28\%     & *         &$5.10*10^6$    &  $1.66*10^6$  & $1.42*10^5$\\
    MySQL42649       & *     & \Yes{1}  & \Yes{1}   & \No         & *       & 8.28\%  &  6.10\%     &  3.93\%     & *         &$7.25*10^3$    &  $1.14*10^3$  &  128.53\\
    MySQL44723       & *     & \Yes{1}  & \Yes{1}   & \Yes{1}     & *       & 7.10\%  &  3.16\%     &  2.24\%     & *         &$3.23*10^5$    &  $1.83*10^5$  & $1.46*10^4$\\
    \midrule
    Apache3278       & \Yes{1} & \No      & \No       & \No         & 0.23\%    & 0.23\%  &   0.23\%    &  0.23\%     & 0.21        & 0.01          & 0             & 0\\
    Apache34464      & \Yes{3} & \Yes{3}  & \Yes{3}   & \Yes{3}     & 29.45\%   & 2.62\%  &   0.18\%    &  0.04\%     & $2.50*10^7$ &$2.50*10^6$    & $2.49*10^5$   &$2.50*10^4$\\
    Apache47223      & \Yes{1} & \Yes{1}  & \Yes{1}   & \Yes{1}     & 12.58\%   & 1.28\%  &   0.13\%    &  0.12\%     & $6.27*10^6$ &$6.26*10^5$    & $6.27*10^4$   &$6.27*10^3$\\
    Apache32546      & \Yes{1} & \Yes{1}  & \Yes{1}   & \Yes{1}     & 0.24\%    & 0.39\%  &   0.38\%    &  0.40\%     & $9.75*10^3$ &977.72         & 99.01         &9.5\\
    \midrule
    GCC1687          & \Yes{1} & \Yes{1}  & \Yes{1}   & \Yes{1}     & 47.30\%   & 5.34\%  &  0.80\%     &   0.43\%    & $3.18*10^7$ &$3.18*10^6$    & $3.18*10^5$   &$3.17*10^4$\\
    GCC8805          & \Yes{4} & \Yes{4}  & \Yes{4}   & \Yes{4}     & 50.92\%   & 7.33\%  &  1.81\%     &   1.05\%    & $1.63*10^7$ &$1.63*10^6$    & $1.63*10^5$   &$1.63*10^4$\\
    GCC15209         & \Yes{1} & \Yes{2}  & \Yes{1}   & \Yes{2}     & 41.06\%   & 8.43\%  &  2.37\%     &   1.27\%    & $3.35*10^4$ &$3.35*10^3$    & 334.72        &33.64\\
    GCC21430         & \Yes{1} & \Yes{1}  & \Yes{1}   & \Yes{1}     & 64.98\%   & 13.68\% &  7.55\%     &   5.07\%    & $9.15*10^7$ &$9.15*10^6$    & $9.15*10^5$   &$9.15*10^4$\\
    GCC46401         & \Yes{2} & \Yes{2}  & \Yes{2}   & \Yes{2}     & 88.97\%   & 13.04\% &  2.91\%     &   0.46\%    & $8.88*10^7$ &$8.88*10^6$    & $8.88*10^5$   &$8.88*10^4$\\
    GCC12322         & \No     & \No      & \No       & \No         & 15.55\%   & 2.33\%  &  2.33\%     &   0.56\%    & $9.97*10^7$ &$9.97*10^6$    & $9.97*10^5$   &$9.97*10^4$\\

    \bottomrule
   \end{tabular}
  }
  \end{adjustwidth}
  %\nocaptionrule
  \caption{Diagnosis capability, overhead, and average number of samples in each run under different sampling rates by using 1000 success/failure runs 
  (*: no results are available, because
   hardware-based sampling cannot be as frequent as $1/100$ and software-based
   CBI sampling does not apply for these C++ benchmarks.
   )}
  \label{tab:rate}
\end{table*}


\paragraph{Run-time overhead}
As shown in Table \ref{tab:LBR}, the run-time overhead is small
under the default sampling rate (i.e., 1 out of 10000).
It is below 5\% in all but three cases, and is always below 8\%.

As expected, the overhead is sensitive with the sampling rate. As shown in
Table~\ref{tab:rate}, it can be further lowered to be mostly below 2\% under
the $\frac{1}{10^{5}}$ sampling rate, and could be as large as over 40\% under
the $\frac{1}{100}$ sampling rate.


\paragraph{Diagnosis capability}
As shown by Table \ref{tab:LBR}, with 1000 success runs
and 1000 failure runs, sampling did very little damage to
the diagnosis capability of statistical debugging. 
Apache\#3278 is the only one, among all benchmarks, where 
failure diagnosis fails under this sampling setting.
For all other benchmarks, the
rankings of the ideal failure predictors remain the same as those 
without sampling in Table \ref{tab:in-house}.

Also as expected, the diagnosis capability would decrease under sparser
sampling or fewer failure/success runs. As shown in Table \ref{tab:rate},
under the default setting of 1000 success/failure runs, the diagnosis capability
is roughly the same between $\frac{1}{10^{3}}$ sampling rate and 
$\frac{1}{10^{4}}$ sampling
rate, but would drop with $\frac{1}{10^{5}}$ sampling rate. 
Four benchmarks that can be diagnosed 
with more frequent sampling cannot be diagnosed with
$\frac{1}{10^{5}}$ sampling rate.
Clearly, more
runs will be needed to restore the diagnosis capability with a lower
sampling rate.

\paragraph{Diagnosis latency}
Diagnosis latency versus run-time overhead and diagnosis capability 
is a fundamental trade-off facing
sampling-based statistical debugging for functional bugs \citep{liblit03,liblit05,CCI}.
With sampling, intuitively, more failure runs are needed
to collect sufficient diagnosis information. This is \textbf{not} a problem for widely
deployed software projects. In those projects, the same failure tends to quickly occur
for many times on many users' machines
\citep{hunt.sosp09}. However, this is a problem
for software that is not widely deployed.

We quantitatively measured the impact of sampling to diagnosis latency in 
Table \ref{tab:LBR}. As we can see, three benchmarks need about 100 failure
runs for their sampling-based diagnosis to produce useful results; 
four benchmarks need about 500 failure runs; and one benchmark, Apache\#3278, 
needs more than 1000 failure runs. This
indicates longer diagnosis latencies than the non-sampling-based diagnosis
evaluated in Section \ref{sec:inhouse}, where only 10 failure runs are used.

Interestingly, there are 11 benchmarks, whose diagnosis latency is \textbf{not}
lengthened by sampling. As shown in Table \ref{tab:LBR}, even with only
10 failure runs, the sampling-based diagnosis still produces good failure
predictors. These are exactly all the 11 benchmarks that $\Delta$LDA model 
suits in Table \ref{tab:in-house}. 
For all these benchmarks,
the rankings are exactly the same with or without
sampling, with just 10 failure runs. Consequently, sampling
allows us to achieve low run-time overhead ($<$10\%), high diagnosis capability,
and low diagnosis latency \textbf{simultaneously}, a feat that is almost 
impossible
for sampling based functional bug diagnosis.
%TODO linhai please check if liblit's work did some experiments on this.

The nice results for these 11 benchmarks can be explained by a unique feature
of performance bugs, especially loop-related
performance bugs --- their root-cause related predicates are often evaluated to 
be true for many times in one run, which is why the performance is poor.
Consequently, even under sparse sampling, there is still a high chance that the
root-cause related predicates can be sampled, and be sampled more frequently
than root-cause unrelated predicates.


Finally, even for the other 9 benchmarks, 
$\frac{1}{10^4}$ sampling rate does not 
extend diagnosis
latency by $10^4$ times. In fact, for most of these benchmarks, 100 -- 500
failure runs are sufficient for failure diagnosis under 
$\frac{1}{10^4}$ sampling
rate. Our investigation shows that the root-cause code regions in these 
benchmarks
are all executed for several times during the user-reported failure runs, 
which
is likely part of the reason why users perceived the performance problems. 
Consequently, the negative impact of sampling on diagnosis latency is
alleviated.


