

\begin{table}[h!]
\centering
\scriptsize
\begin{tabular}{@{\hspace{3pt}}l@{\hspace{3pt}}@{\hspace{3pt}}c@{\hspace{3pt}}}
\toprule
Application Suite Description (language) & \# Bugs \\
\midrule
\bigstrut[t]                           
{\bf Apache Suite} 	 & 16\\
%\cline{1-1}
{HTTPD:	Web Server (C)	}& \\
{TomCat:  Web Application Server (Java)}& \\
{Ant:	Build management utility (Java)}& \\
%\hline
%JMeter	& Load test utility (Java) & \\
\midrule                            
{\bf Chromium Suite} Google Chrome browser (C/C++) & 5\\
\midrule
%\multicolumn{2}{|l|}
{\bf GCC Suite}  GCC \& G++ Compiler (C/C++)     & 9\\
\midrule
{\bf Mozilla Suite}  & 19\\
%\cline{1-1}
{Firefox: Web Browser (C++, JavaScript)}& 	\\
{Thunderbird: Email Client (C++, JavaScript)}& \\
\midrule
{\bf MySQL Suite}     & 16	\\
%\cline{1-1}
{Server: Database Server (C/C++)}&  	\\
%\cline{1}
{Connector: DB Client Libraries (C/C++/Java/.Net)} &  	\\
\midrule
{\bf Total}	   & 65 \\
\bottomrule
\end{tabular}
\caption{Applications and bugs used in the study.}
\label{tab:5_app_bug}
\end{table}

The performance problems under this study include all user-reported
performance problems from the benchmark suite 
collected in Chapter~\ref{chap:study}. 
We cannot directly use the baseline benchmark suite, because it contains
bugs that are discovered by developers themselves through code inspection, a
scenario that performance diagnosis does not apply.
Consequently, we carefully read through all the bug reports and identify all 
the \textbf{65} bugs that are clearly reported by users.
These 65 bug reports all contain detailed information about how each 
performance problem is observed by a user and gets diagnosed by developers.
They are the target of the following characteristics study, and will be 
referred to as \textit{user-reported performance problems} or 
simply \textit{performance problems} in the remainder of this chapter.
The detailed distribution of these 65 bugs is shown in Table~\ref{tab:5_app_bug}.


\paragraph{Caveats} 
Similar with all previous characteristics studies, our findings and 
conclusions need to be considered with our methodology in mind. 
As discussed in Chapter~\ref{chap:study}, the bugs in our study are collected 
from representative applications without bias.
We have followed users and developers' discussion to decide what are 
performance problems that are noticed and reported by users, and finally
diagnosed and fixed by developers.
We did not intentionally ignore any aspect of performance problems. 
Of course, our study does not cover performance problems that are 
not reported to or fixed in the bug databases. It also does not cover
performance problems that are indeed reported by users but have undocumented
discovery and diagnosis histories.
Unfortunately, there is no conceivable way to solve these problems.
We believe the bugs in our study provide a representative sample of the 
well-documented fixed
performance bugs that are reported by users in representative applications.


