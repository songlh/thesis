\section{Conclusion}
\label{sec:con}
Software design and implementation defects lead to not only functional 
misbehavior but also performance losses. Diagnosing performance problems
caused by software defects are both important and challenging. This paper 
made several contributions to improving the state of the art of diagnosing
real-world performance problems. Our empirical study showed that end users
often use comparison-based methods to observe and report performance problems,
making statistical debugging a promising choice for performance diagnosis. 
Our investigation of different design points of statistical debugging shows
that branch predicates, with the help of two types of statistical models, are  
especially helpful for performance diagnosis. It points out useful failure
predictors for 19 out of 20 real-world performance problems. 
Furthermore, our investigation shows that statistical debugging can also
work for production-run performance diagnosis with sampling support, incurring 
less than 10\% overhead in our evaluation. Our study also points out
directions for future work on fine-granularity performance diagnosis.


%\acks
%We thank the anonymous reviewers for their valuable comments which have substantially improved the content and presentation of this paper; 
%Professor Darko Marinov and Professor Ben Liblit for their insightful feedback and help; Jie Liu for his tremendous help in statistical concepts and methods.
%This work is supported in part by NSF grants CCF-1018180,
%CCF-1054616, and CCF-1217582; and a Clare Boothe Luce
%faculty fellowship.
  
