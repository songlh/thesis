\section{Understanding Real-World Performance Problem Reporting and Diagnosis}
\label{sec:5_study}

This section aims to understand the performance diagnosis process
in real world. Specifically, we will focus on these two aspects of
performance diagnosis.

\begin{enumerate}
\item How users notice and report performance problems.
This will help us understand the feasibility of applying statistical debugging
to real-world performance problems, as discussed in 
Section~\ref{sec:5_canwe}. Particularly, we will study how users
tell success runs from failure
runs in the context of performance bugs and how to obtain
success-run inputs (i.e., good inputs) and failure-run inputs
(i.e., bad inputs) for performance diagnosis.
\item How developers diagnose performance problems.
This will help us understand the state of practice of performance diagnosis.
\end{enumerate}

\subsection{Methodology}



\begin{table}[h!]
\centering
\scriptsize
\begin{tabular}{@{\hspace{3pt}}l@{\hspace{3pt}}@{\hspace{3pt}}c@{\hspace{3pt}}}
\toprule
Application Suite Description (language) & \# Bugs \\
\midrule
\bigstrut[t]                           
{\bf Apache Suite} 	 & 16\\
%\cline{1-1}
{HTTPD:	Web Server (C)	}& \\
{TomCat:  Web Application Server (Java)}& \\
{Ant:	Build management utility (Java)}& \\
%\hline
%JMeter	& Load test utility (Java) & \\
\midrule                            
{\bf Chromium Suite} Google Chrome browser (C/C++) & 5\\
\midrule
%\multicolumn{2}{|l|}
{\bf GCC Suite}  GCC \& G++ Compiler (C/C++)     & 9\\
\midrule
{\bf Mozilla Suite}  & 19\\
%\cline{1-1}
{Firefox: Web Browser (C++, JavaScript)}& 	\\
{Thunderbird: Email Client (C++, JavaScript)}& \\
\midrule
{\bf MySQL Suite}     & 16	\\
%\cline{1-1}
{Server: Database Server (C/C++)}&  	\\
%\cline{1}
{Connector: DB Client Libraries (C/C++/Java/.Net)} &  	\\
\midrule
{\bf Total}	   & 65 \\
\bottomrule
\end{tabular}
\caption{Applications and bugs used in the study}
\label{tab:app_bug}
\end{table}

\begin{table*}[tb!]
\begin{adjustwidth}{-1.5in}{-1.5in}
\scriptsize
\centering
{
\begin{tabular}{lcccccc}
\toprule
\multicolumn{1}{c}{Categories} &Apache&Chrome&GCC&Mozilla&MySQL&Total\\
\midrule
\multicolumn{1}{l}{\textbf{Comparison within one code base}}
&9&3&7&7&12&38\\
\ \ Comparing the same input with different configurations &2&1&1&1&5&10\\
\ \ Comparing inputs with different sizes&6&2&4&4&6&22\\
\ \ Comparing inputs with slightly different functionality&2&0&3&2&4&11\\
\midrule
\multicolumn{1}{l}{\textbf{Comparison cross multiple code bases}}
&7&3&8&5&4&27\\
\ \ Comparing the same input under same application's different versions
&4&2&8&3&3&20\\
\ \ Comparing the same input under different applications
&4&1&1&2&1&9\\
\midrule
\multicolumn{1}{l}{\textbf{Not using comparison-based methods}}
&3&1&0&9&1&14\\
\bottomrule
\end{tabular}
}
\end{adjustwidth}
\caption{How performance problems are observed by end users (There are overlaps among
    different comparison-based categories; there is no overlap between non-comparison
    and comparison-based categories)}
\label{tab:cmp}
\end{table*}

The performance problems under this study include all user-reported
performance problems from a real-world performance-bug benchmark suite 
collected by 
previous work~\citep{PerfBug}. We briefly discuss this baseline benchmark
suite and our refinement below.

The baseline benchmarks~\citep{PerfBug} contain 110 fixed real-world performance
bugs randomly sampled from five representative open-source software suites.
These five software suites are all large and mature,
with millions lines of codes and well maintained bug databases. 
They also provide a good coverage of different types of software projects, as
shown in Table~\ref{tab:app_bug}.
The 110 bugs contained in this baseline suite are from
on-line bug databases and are tagged by developers as performance
bugs.

We cannot directly use this baseline benchmark suite, because it contains
bugs that are discovered by developers themselves through code inspection, a
scenario that performance diagnosis does not apply.
Consequently, we carefully read through all the bug reports and identify all 
the \textbf{65} bugs that are clearly reported by users.
These 65 bug reports all contain detailed information about how each 
performance problem is observed by a user and gets diagnosed by developers.
They are the target of the following characteristics study, and will be 
referred to as \textit{user-reported performance problems} or 
simply \textit{performance problems} in the remainder of this paper.
The detailed distribution of these 65 bugs is shown in Table~\ref{tab:app_bug}.




\paragraph{Caveats} 
Similar with all previous characteristics studies, our findings and 
conclusions need to be considered with our methodology in mind. 
The applications in our study cover a variety of important software categories, 
workload, development background, and programming languages. However, there are
still uncovered categories, such as scientific computing software and 
distributed systems.

The bugs in our study are collected from an earlier benchmark suite 
\citep{PerfBug} without bias. 
We have followed users and developers' discussion to decide what are 
performance problems that are noticed and reported by users, and finally
diagnosed and fixed by developers.
We did not intentionally ignore any aspect of performance problems. 
Of course, our study does not cover performance problems that are 
not reported to or fixed in the bug databases. It also does not cover
performance problems that are indeed reported by users but have undocumented
discovery and diagnosis histories.
Unfortunately, there is no conceivable way to solve these problems.
We believe the bugs in our study provide a representative sample of the 
well-documented fixed
performance bugs that are reported by users in representative applications.


\begin{table*}[tb!]
\begin{adjustwidth}{-1.5in}{-1.5in}
\scriptsize
\centering
{
\begin{tabular}{lcccccc}
\toprule
\multicolumn{1}{c}{Categories} &Apache&Chrome&GCC&Mozilla&MySQL&Total\\
\midrule
\multicolumn{1}{l}{\textbf{Comparison within one code base}}
&9&3&7&7&12&38\\
\ \ Comparing the same input with different configurations &2&1&1&1&5&10\\
\ \ Comparing inputs with different sizes&6&2&4&4&6&22\\
\ \ Comparing inputs with slightly different functionality&2&0&3&2&4&11\\
\midrule
\multicolumn{1}{l}{\textbf{Comparison cross multiple code bases}}
&7&3&8&5&4&27\\
\ \ Comparing the same input under same application's different versions
&4&2&8&3&3&20\\
\ \ Comparing the same input under different applications
&4&1&1&2&1&9\\
\midrule
\multicolumn{1}{l}{\textbf{Not using comparison-based methods}}
&3&1&0&9&1&14\\
\bottomrule
\end{tabular}
}
\end{adjustwidth}
\caption{How performance problems are observed by end users (There are overlaps among
    different comparison-based categories; there is no overlap between non-comparison
    and comparison-based categories.).}
\label{tab:5_cmp}
\end{table*}







\subsection{How users report performance problems}
In general, to conduct software failure diagnosis, it is critical to understand 
what are the failure symptoms and what information is available for failure
diagnosis. 
Specifically, as discussed in Section~\ref{sec:5_canwe}, to understand the
feasibility of applying statistical debugging for performance diagnosis, we
will investigate two issues: (1)
How do users judge whether a slow execution is caused by large workload or
inefficient implementation, telling success runs from failure
runs?
(2)
What information do users provide to convince developers that inefficient
implementation exists and hence help the performance diagnosis?

\paragraph{How are performance problems observed?}

As shown in Table~\ref{tab:5_cmp}, the majority (51 out of 65) of user-reported 
performance problems are observed through comparison, including
comparisons within one software code base and comparisons across multiple code bases.

\underline{\it Comparison within one code base} 
is the most common way to reveal performance problems.  
In about 60\% of cases, 
users notice huge performance differences among
similar inputs and hence file bug reports.

Sometimes, the inputs under comparison have the same functionality but different
sizes. For example, MySQL\#44723 is reported when users observe that inserting
11 rows of data for 9 times is two times slower than inserting 9 rows of data
for 11 times. As another example, Mozilla\#104328 is reported when users observe
a super-linear performance degradation of the web-browser start-up time in terms
of the number of bookmarks.

Sometimes, the inputs under comparison are doing slightly different tasks.
For example, when reporting Mozilla\#499447, the user mentions that changing the width
of Firefox window, with a specific webpage open, takes a lot of time (a bad input), yet
changing the height of Firefox window, with the same webpage,
takes little time (a good 
input).

%Finally, significantly different performances under the same input and different 
Finally, large performance difference under the same input and different
configurations is also a common reason for users to file bug reports.
For example, when reporting GCC\#34400, the user compared the compilation time
of the same file under two slightly different GCC configurations.
The only difference between these two configurations is that the ``ZCX\_By\_Default''
entry in the configuration file is switched from True to False. 
However, the compilation times goes from 4 seconds to almost 300 minutes.

\underline{\it Comparison across different code bases} 
In about 40\% of the performance problems that we studied, users support
their performance suspicion through a comparison 
across different code bases. For example, GCC\#12322 bug report mentions
that ``GCC-3.3 compiles this file in about five minutes; GCC-3.4 takes
30 or more minutes''. As another example, Mozilla\#515287 bug report
mentions that the same Gmail instance leads to 15--20\% CPU utilization
in Mozilla Firefox and only 1.5\% CPU utilization in Safari.

Note that, the above two comparison approaches do not exclude each other.
In 14 out of 27 cases, comparison results across multiple code bases are reported
together with comparison results within one code base.

\underline{\it Non-comparison based}
For about 20\% of user-reported performance problems, users observe an
absolutely non-tolerable performance and file the bug report without any comparison.
For example, Mozilla\#299742 is reported as the web-browser frozed to crawl.

\paragraph{What information is provided for diagnosis?}

\begin{table*}[tb!]
\begin{adjustwidth}{-.5in}{-.5in}
\scriptsize
\centering
{
\begin{tabular}{lcccccc}
\toprule
&Apache&Chrome&GCC&Mozilla&MySQL&Total\\
\midrule
Total \# of bug reports & 16 & 5 & 9 & 19 & 16 & 65 \\
\midrule
\multicolumn{7}{c}{\bf \# of bad inputs provided}\\
\multicolumn{1}{l}{{\bf 0/?}: No bad input }
&0&0&0&0&0&0\\
\multicolumn{1}{l}{{\bf 1/?}: One bad input}
&0&1&5&6&7&19\\
\multicolumn{1}{l}{{\bf n/?}: A set of bad inputs}
&16&4&4&13&9&46\\
\midrule
\multicolumn{7}{c}{\bf \# of good inputs}\\
\multicolumn{1}{l}{{\bf ?/0}: No good input}
&7&2&2&12&4&27\\
\multicolumn{1}{l}{{\bf ?/1}: One good input}
&0&0&3&0&3&6\\
\multicolumn{1}{l}{{\bf ?/n}: A set of good inputs}
&9&3&4&7&9&32\\
\bottomrule
\end{tabular}
}
\end{adjustwidth}
\caption{Inputs provided in users' bug reports ($n$: 
developers provide a way to generate a large number of inputs.).}
\label{tab:5_input}
\end{table*}

The most useful information provided by users include failure
symptom (discussed above), bad inputs, and good inputs. Here, we refer to the 
inputs that lead to user-observed performance problems
as \textit{bad inputs}; we refer to the
inputs that look similar with some bad inputs but lead to good performance,
according to the users,
as \textit{good inputs}.

\underline{\it Bad inputs} Not surprisingly, users provide problem-triggering
inputs in all the 65 cases. What is interesting is that in about 70\% of
cases (46 out of 65), users describe a category of inputs, instead of just
one input, that can trigger
the performance problem, as shown in Table \ref{tab:5_input}. For example,
in MySQL\#26527, the user describes that loading data from file into partitioned
table can trigger the performance problem, no matter what is the content or 
schema of the table. 

\underline{\it Good inputs} Interestingly, good inputs are specified in almost
60\% of bug reports, as shown in Table \ref{tab:5_input}. 
That is, users describe inputs that look similar with the
bad inputs but have much better performance in all the 38 bug reports
where ``comparison within one code base'' is used to observe the performance
problem.
Furthermore,
in 32 bug reports, users describe how to generate a large number of good
inputs, instead of just one good input.
For example, when reporting MySQL\#42649, the user
describes that executing queries on tables using the default charset setting or
the \textit{latin1} charset setting (good inputs) will not cause lock contention, while queries
on tables using other types of charset settings (bad inputs) may cause lock contention.
Note that, this is much rarer in functional bug reports, which is why special
tools are
designed to automatically generate inputs that execute correctly
and are similar with bad inputs, when diagnosing functional-bug failures~\citep{delta}.

%\paragraph{Hypothesis Testing}

%We conduct several hypothesis testings to evaluate whether we have enough evidence to draw conclusions about performance bugs based on our sample. We choose 0.01 as significance level. 
%Under this setting, if we draw a conclusion, the conclusion only has 1\% probability to be wrong. 
%Our testing results show that more performance bugs are reported with a set of bad inputs, 
%but we fail to draw a conclusion that more performance bugs are reported with comparison-based methods. 


\subsection{How developers diagnose performance problems}

To collect the diagnosis time, we check the bug databases and calculate the
time between a bug report being posted and a correct fix being proposed.
Of course, strictly speaking, this time period can be further broken down to
bug-report assignment, root-cause locating, patch design, and so on. 
Unfortunately, we cannot obtain such fine-grained information accurately
from the databases. Most Apache, Chrome, and MySQL bugs in
our study do not have clear assignment time in record. For GCC bugs in
study, report assignment takes about 1\% of the overall diagnosis
time on average; for Mozilla bugs in study, report assignment takes about
19\% of the overall diagnosis time on average.

Our study shows that it takes 129 days on average for developers to finish
diagnosing a performance problem reported by users.
Among the 5 software projects, the Chrome project has the shortest average
performance-diagnosis time (59 days), and Apache project has the longest
average diagnosis time (194 days).
Comparing with the numbers reported in Chapter~\ref{chap:study}, 
the time to diagnose user-reported performance problems is slightly shorter
than that for non-user-reported performance problems,
and similar or longer than that of 
functional bugs. 

We also studied how developers diagnose performance problems.
The only type of diagnosis tools that are mentioned in bug reports are
performance profilers. They are mentioned in 13 out of the 65 reports.
However, even after the profiling results are provided, it still takes
developers 116 days on average to figure out the patches.


\subsection{Implications of the study}
\label{sec:5_study_imp}

\ \ \underline{\textit{Implication 1}}
Performance bugs and functional bugs are observed in different ways.
Intuitively, the symptoms of many functional bugs, such as assertion violations,
error messages, and crashes, can be easily identified by looking
at the failure run alone~\citep{LiASID06}.
In contrast, the manifestation
of performance bugs often gets noticed through comparison.
%TODO check functional bugs and conduct a statistical test here
We have randomly sampled 65 user-reported functional bugs from the same set
of applications (i.e., Apache, Chrome, GCC, Mozilla, and MySQL) and found that
only 8 of them are observed through comparison.
Statistical Z tests~\citep{ztest} show that the above observation is 
statistically
significant --- at the 99\% confidence level, 
a user-reported performance bug is more likely to be observed through 
comparison than a user-reported functional bug.

\underline{\textit{Implication 2}}
Although judging execution efficiency based on execution time alone 
is difficult in
general, distinguishing failure runs from success runs and obtaining bad and good
inputs are fairly straightforward based on performance-bug reports filed by 
users.
Our study shows that most user-reported performance problems are observed when 
two sets of similar inputs demonstrate very different performances (38 out of 
65 cases). 
Most of these cases (32 out of 38), users provide explicit good and bad 
input-generation methodology. 
In other cases (27 out of 65),
users observe that an input causes intolerably slow execution or very different
performances across similar code bases. Distinguishing
failure runs from success runs and bad inputs from good inputs are 
straightforward in these cases based on the symptoms described
in the bug reports, such as ``frozed the GUI to
crawl'' in Mozilla\#299742 and 10X more CPU utilization rate than Safari 
under the same input in Mozilla\#515287. 

\underline{\textit{Implication 3}}
Statistical debugging is naturally suitable for diagnosing many
user-reported performance problems,
because most performance bugs are observed by users through comparison and many
performance-bug reports (38 out of 65) already contain information about 
both bad and good inputs that are similar with each other.
Statistical tests~\citep{ztest} show that with 90\% statistical confidence, 
a user-filed performance-bug report is more likely to contain both 
bad and good inputs than not.
Comparing the 65 randomly sampled functional bugs mentioned above with the 65
performance bugs, 
statistical tests~\citep{ztest} show that, at the 99\% confidence level, 
a user-filed performance-bug report is more likely to contain
good inputs than a user-filed
functional-bug report.
Previous statistical debugging work tries hard to generate good
inputs to diagnose functional bugs~\citep{delta}. This task is
likely easier for performance failure diagnosis.

\underline{\textit{Implication 4}}
Developers need tools, in addition to profilers, to diagnose
user-reported performance problems.
