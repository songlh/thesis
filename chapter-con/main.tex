\chapter[Conclusion]{Conclusion}
\label{chap:con}

Performance bugs are software implementation mistakes, which can cause inefficient execution. 
Performance bugs are common and severe, and they have already become one major source of software's performance problem. 
With the increasing complexity
of modern software and workload, the meager increases of single-core hardware performance, and the pressing energy concerns, 
it is urgent to combat performance bugs. 

This dissertation targets to improve the state-of-the-art performance bug fighting techniques. 
We start from an empirical study on real-world performance bugs in order to get a better understanding of performance bugs in Chapter~\ref{chap:study}.
Inspired by our empirical study, 
we build a series of rule-based static checkers to identify previously unknown performance bugs in Chapter~\ref{chap:detec}. 
We explore how to apply statistical debugging to diagnose user-perceived performance bugs in Chapter~\ref{chap:sd}. 
Statistical debugging cannot answer all the questions, 
so we design a series of static-dynamic hybrid analysis for inefficient 
loops to provide fine-grained diagnosis information in Chapter~\ref{chap:ldoctor}. 

In this chapter, we first summarize our study work and built techniques in Section~\ref{sec:7_summary}. 
We then list a set of lessons learned over the course of our work in Section~\ref{sec:7_lessons}. 
Finally we discuss directions for future research in Section~\ref{sec:7_future}.

\section{Summary}
\label{sec:7_summary}
This dissertation can be divided into three parts: 
performance bug understanding, performance bug detection, and performance failure diagnosis. 
We now summary each part in turn. 

\subsection{Performance bug understanding}

Like functional bugs, research on performance bugs should also be guided by empirical studies. 
Poor understanding of performance bugs is part of causes of today's performance-bug problem. 
In order to improve the understanding of real-world performance bugs, 
we conduct the first empirical study on 110 real-world performance bugs 
randomly sampled from 5 open-source software suites (Apache, Chrome, GCC, Mozilla and MySQL).  

Following the lifetime of performance bugs, our study is mainly performed in 4 dimensions. 
We study the root causes of performance bugs, how they are introduced, how to expose them and how to fix them. 
The main findings of our study include: 
(1) performance bugs have dominating root causes and fix strategies, 
which are highly correlated with each other; 
(2) workload mismatch and misunderstanding of API's performance features 
are two major reasons why performance bugs are introduced; 
(3) around half of studied performance bugs need inputs with both special features 
and large scales to manifest. 


Our empirical study can guide future research on performance bugs, 
and it has already motivated our own bug detection and diagnosis projects.

\subsection{Performance bug detection}

Inspired by our empirical study, 
we hypothesize that efficiency rules widely exist in software, 
rule-violations can be statically checked, and violations also widely exist. 
To test our hypothesis, we manually inspect final patches of fixed performance bugs from
Apache, Mozilla and MySQL in our studied performance-bug set. 
We extract efficiency rules from 25 bug patches, and implement static checkers for these rules. 

In total, our static checkers find 332 previously unknown Potential Performance Problems (PPPs) 
from latest versions of Apache, Mozilla and MySQL. 
Among them, 101 are inherit from original buggy versions where final patches are applied. 
Tools are needed to help developers automatically and systematically find all similar bugs.
12 PPPs are introduced later. Tools are needed to help developers avoid making the same mistakes repeatedly. 
219 PPPs are found by cross-application checking. 
There are generic rules among different software.  
Our experimental results verify all our hypothesis. 
Rule-based performance-bug detection is a promising direction. 

\subsection{Performance failure diagnosis}

Due to the preliminary tool support, many performance bugs escape from in-house performance testing and manifest in front of end users. 
After users report performance bugs, developers need to diagnose them and fix them.
Diagnosing user-reported performance failure is another key aspect of fighting performance bugs. 

We investigate the feasibility and design space to apply statistical debugging to performance failure diagnosis.
After studying 65 user-reported performance bugs in our bug set, 
we find that majority of performance bugs are observed through comparison, 
and many user-file performance-bug reports contain not only bad inputs, but also similar and good inputs.
Statistical debugging is a natural fix for user-reported performance bugs. 
We evaluate three types of widely used predicates and two representative statistical models. 
Our evaluation results show that branch predicate plus two statistical models can effectively diagnose user-reported performance failure. 
Basic model can help diagnose performance failure caused by wrong branch decision, and $\Delta $LDA model can identify inefficient loops.  
We apply sampling to performance failure diagnosis. Our experimental results show that
special nature of loop-related performance bugs allows sampling to lower runtime overhead without sacrificing diagnosis latency, 
and this is very different from functional failure diagnosis.

We build LDoctor to further provide fine-grained diagnosis information for inefficient loops through two steps. 
We first figure out a root-cause taxonomy for common inefficient loops through a comprehensive study on 45 inefficient loops. 
Our taxonomy contains two major categories: resultless and redundancy, and several subcategories. 
Guided by our taxonomy, we then design a series of analysis for inefficient loops. 
Our analysis 
focuses its checking on suspicious loops pointed out by statistical debugging, 
hybridize static and dynamic analysis to balance accuracy and performance, 
and relies on sampling and other designed optimization to further lower runtime overhead. 
We evaluate LDoctor by using 18 real-world inefficient loops. 
Evaluation results show that LDoctor can cover most root-cause subcategories, 
report few false positives, and bring a low runtime overhead. 

\section{Lessons Learned}
\label{sec:7_lessons}

In this section, we present a list of general lessons we learned while working on this dissertation.

%\begin{itemize}

%\item 
{\bf Identifying performance failure is hard.}
Performance bugs do not have fail-stop symptoms. 
Both developers and end users face challenges in identifying performance failure runs. 
Performance bugs tend to hide longer in software than functional bugs (Section~\ref{sec:3_other}).
Some performance bugs, which can be triggered by almost all inputs, can still escape from in-house performance testing (Section~\ref{sec:3_exp}). 
Fearing that described symptoms are not convincing enough, 
end users sometimes use more than one 
comparison-based methods when they file performance-bug reports (Section~\ref{sec:5_study}). 
Techniques targeting automatically identifying performance failure runs, 
like automated oracles for performance bugs and performance assertion, are solely needed. 

%\item 
{\bf Similar mistakes can be made everywhere. }
Implementation mistakes are usually caused by developers' misunderstanding of programming languages, API, workload, documents and so on. 
Before correcting the misunderstanding, 
developers definitely would make similar mistakes in other places.
Misunderstanding could also be shared among different developers, and same mistakes could also be made by more than one developers.
When fixing one bug, it is necessary to systematically check software to find similar bugs and fix them altogether. 

%\item 
{\bf Non-buggy codes can become buggy.}
Some performance bugs in our benchmark set are not born buggy. 
They become performance bugs, due to workload shift, code changes in other part of software, or hardware changes. 
Periodical workload study, performance change impact analysis, 
and systematically profiling when porting software to new hardware are needed during software development and maintenance.

%\item 
{\bf Static analysis is not accurate enough.}
Our experience shows that static analysis is not good enough when analyzing performance bugs, 
because some codes are inefficient only under certain workload. 
In Chapter~\ref{chap:detec}, accurately checking some efficiency rules relies on runtime or workload information. 
Our detection techniques are only based on static checking, and this is one reason why we have false positives. 
In Chapter~\ref{chap:ldoctor}, static analysis can only identify static resultless types and conduct slicing. 
We need dynamic information about portions of resultless iterations and values of source instructions, 
so we build LDoctor based on static-dynamic hybrid approaches. 


%\item {\bf Sampling is suitable to combat performance bugs. }
%Sampling can help lower runtime overhead while recording dynamic information. 
%Because of performance bugs' repetitive patterns, 
%sampling does not hurt latency as much as functional bugs. 
%In Chapter~\ref{chap:sd} and Chapter~\ref{chap:ldoctor}, 
%we show that how sampling help statistical debugging and LDoctor. 
%We believe that sampling will help other in-house performance bug detection and diagnosis techniques, 
%and make them applicable in production runs. 



%\end{itemize}

\section{Future Work}
\label{sec:7_future}

Future work can explore how to combat performance bugs through different aspects from this thesis. 
The following are several potential research opportunities:


%\begin{itemize}

{\bf On-line workload monitoring.} 
Our empirical study in Chapter~\ref{chap:study} 
shows that many performance bugs are introduced due to developers' misunderstanding of workload in reality. 
We think that monitoring workload from productions runs is a possible solution. 
Production-run techniques need to keep a low runtime overhead.
How to collect accurate workload information from deployed software in a low overhead remains an open issue. 

{\bf Improving accuracy of static analysis by leveraging existing dynamic information.}
One lesson we learned is that static analysis alone is not accurate enough, 
and dynamic information is needed (Section~\ref{sec:7_lessons}). 
Systems may have already recorded some dynamic information through logging or tracing.
How to leverage existing dynamic information to improve the results of static analysis remains an open issue. 

{\bf Applying sampling to existing in-house performance-bug techniques.}
Sampling can help lower runtime overhead, while recording dynamic information. 
Due to performance bugs' repetitive patterns, 
sampling does not hurt latency as much as functional bugs. 
In Chapter~\ref{chap:sd} and Chapter~\ref{chap:ldoctor}, 
we show that how sampling help statistical debugging and LDoctor. 
How to leverage sampling to make other existing in-house performance-bug detection 
and performance failure diagnosis techniques applicable in production runs remains an open issue. 


{\bf Test input generation for performance bugs.}
In Chapter~\ref{chap:study}, 
we discuss that almost half of studied performance bugs need inputs with both special features and large scales to manifest. 
Existing techniques are designed to generate inputs with good code coverage and focus only on special features.
How to extend existing input-generation techniques with emphasis on large scales remains an open issue. 
Another important problem during performance testing is to automatically judge whether a problem bug has occurred. 
How to leverage existing dynamic performance-bug detection techniques to build test oracles for performance bugs also remains an open issue.  


{\bf Performance-aware annotation system.}
Our study in Chapter~\ref{chap:study} shows that performance-aware annotations, 
which can help developers maintain and communicate APIs' performance features, 
can help avoid performance bugs. 
How to automatically identify performance features, 
like existence of lock or IO, 
through program analysis or document mining on existing software, remain an open issue. 

%\end{itemize} 

\section{Closing Words}
Performance bugs are software implementation mistakes, which can slow down the program. 
With new software and hardware trends and pressing energy concerns, it is very important to fight performance bugs. 
We hope that this dissertation can help researchers and developers 
by providing better understanding of real-world performance bugs. 
We also hope that this dissertation can provide hints 
and implications for developers when they try to identify previously unknown performance bugs 
and diagnose user-perceived performance failure. 

