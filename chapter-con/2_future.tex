\section{Future Work}
\label{sec:7_future}

Future work can explore how to combat performance bugs through different aspects from this thesis. 
The following are several potential research opportunities:


%\begin{itemize}

{\bf On-line workload monitoring.} 
Our empirical study in Chapter~\ref{chap:study} 
shows that many performance bugs are introduced due to developers' misunderstanding of workload in reality. 
We think that monitoring workload from productions runs is a possible solution. 
Production-run techniques need to keep a low runtime overhead.
How to collect accurate workload information from deployed software in a low overhead remains an open issue. 

{\bf Improving accuracy of static analysis by leveraging existing dynamic information.}
One lesson we learned is that static analysis alone is not accurate enough, 
and dynamic information is needed (Section~\ref{sec:7_lessons}). 
Systems may have already recorded some dynamic information through logging or tracing.
How to leverage existing dynamic information to improve the results of static analysis remains an open issue. 

{\bf Applying sampling to existing in-house performance-bug techniques.}
Sampling can help lower runtime overhead, while recording dynamic information. 
Due to performance bugs' repetitive patterns, 
sampling does not hurt latency as much as functional bugs. 
In Chapter~\ref{chap:sd} and Chapter~\ref{chap:ldoctor}, 
we show that how sampling help statistical debugging and LDoctor. 
How to leverage sampling to make other existing in-house performance-bug detection 
and performance failure diagnosis techniques applicable in production runs remains an open issue. 


{\bf Test input generation for performance bugs.}
In Chapter~\ref{chap:study}, 
we discuss that almost half of studied performance bugs need inputs with both special features and large scales to manifest. 
Existing techniques are designed to generate inputs with good code coverage and focus only on special features.
How to extend existing input-generation techniques with emphasis on large scales remains an open issue. 
Another important problem during performance testing is to automatically judge whether a problem bug has occurred. 
How to leverage existing dynamic performance-bug detection techniques to build test oracles for performance bugs also remains an open issue.  


{\bf Performance-aware annotation system.}
Our study in Chapter~\ref{chap:study} shows that performance-aware annotations, 
which can help developers maintain and communicate APIs' performance features, 
can help avoid performance bugs. 
How to automatically identify performance features, 
like existence of lock or IO, 
through program analysis or document mining on existing software, remain an open issue. 

%\end{itemize} 

\section{Closing Words}
Performance bugs are software implementation mistakes, which can slow down the program. 
With new software and hardware trends and pressing energy concerns, it is very important to fight performance bugs. 
We hope that this dissertation can help researchers and developers 
by providing better understanding of real-world performance bugs. 
We also hope that this dissertation can provide hints 
and implications for developers when they try to identify previously unknown performance bugs 
and diagnose user-perceived performance failure. 
