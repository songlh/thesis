\chapter[Introduction]{Introduction}
\label{chap:introduction}

We all want software to run fast and efficiently. 
Software performance severely affects usability of software system 
and is one of the most important problems in computer science research. 
Performance bugs are one major source of software's slowness and inefficiency. 
Due to their non fail-stop symptoms, 
performance bugs are easy to escape from in-house testing and difficult to be diagnosed. 
Nowadays, the urgency to address performance bugs is becoming even more important 
with new hardware and software trends and increasing concerns about energy constraints. 

Facing the challenge of performance bugs, 
this dissertation proposes effective performance bug detection 
and performance bug diagnosis approaches based on a 
comprehensive characteristics study of real-world performance bugs.

\section{Motivation}
%TODO add dick site's talk. Google's need for pattern. Solve 99% tail
%worth millions of dollars; finding a couple of perf bugs in the late stage
% of development needs several months of several experts' time!
Slow and inefficient software can easily frustrate users and
cause financial losses.
%\cite{responsetimeibm}.
Although researchers have devoted decades to
transparently improving software performance,
{\it performance bugs} continue to pervasively
degrade performance
and waste computation resources in the field~\citep{rily.perftest}.
%developers still frequently use inefficient code sequences, which
%could be fixed by simple patches, that cause significant 
%performance degradation and computation waste in the field
%(i.e., {\bf performance bugs}). 
Meanwhile, current support for combating performance bugs is preliminary due
to the poor understanding of real-world performance bugs.
%To address this problem, an improved understanding of real-world 
%performance bugs is sorely needed.

Following the convention of developers and researchers 
on this topic~\citep{s2e,perf.fse10,rily.perftest,perfantipattern},
we refer to performance bugs as software defects where
relatively simple {\it source-code} changes can significantly speed up
software, while preserving functionality.
These defects can{\bf not} be optimized away by state-of-practice compilers,
thus bothering end users.

\begin{figure}
\codefig{Apache45464}
\caption{A performance bug from Apache-HTTPD
  {(`+' and `-' denote the 
    code added and deleted to fix this bug)}}
\label{fig:Apache45464}
\end{figure}


Figure~\ref{fig:Apache45464} shows an example of a real-world performance bug. 
Apache HTTPD developers forgot to change a parameter of API \Code{apr_stat} after an API upgrade. 
This mistake causes \Code{apr_stat} to retrieve more than necessary information 
from the file system and leads to more than ten times slowdown in Apache server. 
After changing to \Code{APR_FINFO_TYPE}, \Code{apr_stat} will retrieve exactly what 
developers originally needed through \Code{APR_FINFO_NORM}. 


Performance bugs exist widely in released software. 
For example, Mozilla developers have fixed 5--60 performance bugs
reported by users {\it every month} over the past 10 years.
The prevalence of performance bugs is inevitable because
little work has been done to help developers avoid
performance-related mistakes. In addition,
performance testing mainly relies on ineffective black-box random 
testing and manual input design, which allows the majority of performance bugs
to escape~\citep{rily.perftest}.

Performance bugs lead to reduced throughput, increased latency, and 
wasted resources in the field. 
In the past, they have caused several 
highly publicized failures, causing hundred-million dollar software 
projects to be abandoned~\citep{colorado,uk}.

Worse still, performance problems are costly to 
diagnose due to their non fail-stop symptoms.
Software companies may need several months of effort by experts
to find a couple of performance bugs that cause a few 
hundred-millisecond delay in
the 99th percentile latency of their service~\citep{dicksites}.

The following trends will make the performance-bug problem more critical
in the future:

%They have not been well studied before and worth our research attention.

{\bf Hardware:} For many years, 
  %the single threaded interpretation of 
  Moore's law ensured that hardware would make software faster over time 
  with no software
  development effort. In the multi-core era, when each core is unlikely to 
  become faster, performance bugs are particularly harmful.

{\bf Software:} The increasing complexity of software systems and rapidly 
changing workloads provide new opportunities for performance waste and
new challenges in diagnosis
%Development practices such as code reuse and framework based software
%are making things even worse
\citep{BloatFSE2008}.
Facing the increasing pressure on productivity, 
developers cannnot combat performance bugs without automated tool support. 

{\bf Energy efficiency:} 
Increasing energy costs provide a powerful economic 
argument for avoiding performance bugs. 
%Generating power to run data centers is
%expensive. Storing power for use in mobile devices is also costly. 
When one is willing to sacrifice the service quality to reduce 
energy consumption~\citep{green.pldi10,asplos11karthik}, 
ignoring performance bugs is unforgivable.
For example, by fixing bugs that have doubled the
execution time,
one may potentially
halve the carbon footprint of buying and operating computers.

Performance bugs may not have been reported as often as functional bugs, because 
they do not cause fail-stop failures.
However, considering the preliminary support for
combating performance bugs,
it is time to pay more attention to them
when we enter a new resource-constrained computing world.
