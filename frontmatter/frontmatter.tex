\svnidlong{$LastChangedBy$}{$LastChangedRevision$}{$LastChangedDate$}{$HeadURL: http://freevariable.com/dissertation/branches/diss-template/frontmatter/frontmatter.tex $}
\vcinfo{}

%%% SOME OF THIS CODE IS ADAPTED FROM THE VENERABLE withesis.cls

% COPYRIGHT PAGE
%  - To include a copyright page use \copyrightpage
\copyrightpage

% DEDICATION
\begin{dedication}
	\emph{To my parents, Zuoyu Song and Cuihua Fu.}
\end{dedication}

%% BEGIN PAGESTYLE

%%% You can pick a pagestyle if you want; see the memoir class
%%% documentation for more info.  The default ``deposit'' option meets
%%% the UW thesis typesetting requirements but is probably
%%% unsatisfactory for making a version of your dissertation that
%%% won't be deposited to the graduate school (e.g. for web or a nice
%%% printed copy)

\chapterstyle{deposit}
\pagestyle{deposit}


% ACKNOWLEDGMENTS
\begin{acks}
First and foremost, I would like to express my wholehearted gratitude to my advisor, 
Professor Shan Lu, for her generous support and guidance during my PhD. 

The first email I sent to Shan is to ask whether I could provide her name as my potential advisor, when I applied my student visa. 
Shan kindly approved this. 
I am so lucky to begin my independent study with her immediately after I came to US. 
I can still remember the joy in my heart when she told me that she would pick me as her student, after we submit AFix. 

Throughout my PhD, Shan gave me numerous invaluable academic advice, 
ranging from the details of how to present data in my paper to the broad vision of 
how to explore an initial research idea. 
Besides research, 
Shan also teaches me how to better communicate with other people, 
how to manage my time, and how to be confident. 
Shan served as my role model in the past five years. 
I am so fortunate to observe and learn how she learns new knowledge, 
conducts research and succeeds in her career. 
PhD journey is tough. 
I feel extremely thankful for Shan's patience, support and actionable suggestions in each stage. 
I do not think I can get to the destination without her encouragement. 
 
I gratefully thank Professor Ben Liblit and Professor Darko Marinov for their kindly help during and after collaborating. 
Ben and Darko are always encouraging and helpful whenever I encounter research problems. 
Ben also helps me go through all paperwork and payment after Shan left to UChicago. 
I worked in Illinois for around one week, and it is an exciting experience to learn technical skills closely from Darko. 

I would like to thank all other committee members, Professor Remzi Arpaci-Dusseau, 
Professor Aws Albarghouthi, and Professor Xinyu Zhang, for their priceless comments and constructive criticisms on my thesis. 
It is really my honor to have all these professors on my committee. 
I would also thank Professor Loris D'Antoni for coming to my defense and leaving invaluable feedback. 

Thanks also go to every student in Shan's group and fellow students I worked with for 
the opportunity to learn together. 
I worked with Guoliang Jin on my first two research projects, and I want to thank him for helping me through my junior years. 
I want to thank Adrian Nistor for treating me so many lunches and solving so many weird problems I encountered in Illinois.
I want to thank Wei Zhang, Po-Chun Chang, Xiaoming Shi, Dongdong Deng, Rui Gu, Joy Arulraj, 
Joel Sherpelz, Haopeng Liu, and Yuxi Chen for making our group fun and exciting. 

I also want to thank many other people in UWisconsin. 
Thanks to Professor David Page, Professor Michael Swift, Professor Susan Horwitz, 
Professor Jerry Zhu, and Professor Eric Bach for knowledge I learned from their courses. 
Thanks to Angela Thorp for helping me through various administration stuff. 
Thanks to Peter Ohmann for suggestions about how to polish my talk. 
Thanks to my great roommates, Hao Wang, Ce Zhang, Jia Xu, and Huan Wang for giving me such great accompany. 
Thanks to all other good friends, Lanyue Lu, Wenfei Wu, Suli Yang, Yupu Zhang, Yiying Zhang, 
Jie Liu, Wentao Wu, Wenbin Fang, Weiyan Wang, Xiaozhu Meng, Junming Xu, Yimin Tan, Yizheng Chen, Ji Liu, Jun He, and Ao Ma for their help and support in different ways. 

I benefit a lot from my summer intern in NEC Labs America. 
I want to thank the company as well as my mentor Dr. Min Feng, for providing a terrific internship experience. 
 
Last but not least, I want to thank my parents back in China for their unconditional love and trust. 
Whenever I struggle with my PhD life, my parents are always supportive and encouraging. 
My work would not be nearly as meaningful without them. 
No words can express my gratitude and love to them. 
I dedicate the whole thesis to the most important persons in my life.  

\end{acks}

% CONTENTS, TABLES, FIGURES
\renewcommand{\printtoctitle}[1]{\chapter*{#1}}
\renewcommand{\printloftitle}[1]{\chapter*{#1}}
\renewcommand{\printlottitle}[1]{\chapter*{#1}}

\renewcommand{\tocmark}{}
\renewcommand{\lofmark}{}
\renewcommand{\lotmark}{}

\renewcommand{\tocheadstart}{}
\renewcommand{\lofheadstart}{}
\renewcommand{\lotheadstart}{}

\renewcommand{\aftertoctitle}{}
\renewcommand{\afterloftitle}{}
\renewcommand{\afterlottitle}{}

\renewcommand{\cftchapterfont}{\normalfont} 
\renewcommand{\cftsectionfont}{\itshape} 
\renewcommand{\cftchapterpagefont}{\normalfont} 
\renewcommand{\cftchapterpresnum}{\bfseries} 
\renewcommand{\cftchapterleader}{\hskip2mm} 
\renewcommand{\cftsectionleader}{\hskip2mm} 
\renewcommand{\cftchapterafterpnum}{\cftparfillskip} 
\renewcommand{\cftsectionafterpnum}{\cftparfillskip} 

% \captionnamefont{\small\sffamily} 
% \captiontitlefont{\small\sffamily} 

% \renewcommand{\contentsname}{contents}
% \renewcommand{\listfigurename}{list of figures}
% \renewcommand{\listtablename}{list of tables}

\tableofcontents

\clearpage
\listoftables

\clearpage
\listoffigures

%\clearpage
% NOMENCLATURE
 %\begin{conventions}
 %\begin{description}
 % \item{\makebox[0.75in][l]{term}
 %        \parbox[t]{5in}{definition\\}}
 % \end{description}
 %\input{frontmatter/conventions}
 %\end{conventions}

%\advisorname{Gottlob Frege}
%\advisortitle{Professor}
% ABSTRACT
%\begin{umiabstract}
%  Everyone wants software to run fast. 
Slow and inefficient software can easily frustrate end users and cause economic loss. 
Software-inefficiency problem has already caused several highly publicized failures. 
One major source of software's slowness is performance bug.
Performance bugs are software implementation mistakes that can cause inefficient execution. 
Performance bugs cannot be optimized away  by state-of-practice compilers. 
Many of them escape from in-house testing and manifest in front of end users, 
causing severe performance degradation and huge energy waste in the field. 
Performance bugs are becoming more critical, 
with the increasing complexity of modern software and workload,
the meager increases of single-core hardware performance, and the 
pressing energy concerns.
It is urgent to combat performance bugs.

This thesis works on three directions to fight performance bugs: 
performance bug understanding, rule-based performance-bug detection, 
and performance failure diagnosis. 

Building better tools needs a better understanding of performance bugs. 
In order to improve the understanding of performance bugs, 
we randomly sample 110 real-world performance bugs from 
5 large open-source software suites (Apache, Chrome, GCC, Mozilla and MySQL), 
and conduct the first empirical study on performance bugs. 
Our study is mainly performed to understand what are common root causes of performance bugs, 
how performance bugs are introduced, how to expose performance bugs and how to fix performance bugs. 
Important finds include: (1) there are dominating root causes and fix strategies for performance bugs, 
and root causes are highly correlated with fix strategies; 
(2) workload and API issues are two major reasons causing performance bugs to be introduced; 
(3) performance bugs require inputs with both special features and large scales to be exposed effectively. 
Our empirical study can guide future research on performance bugs, 
and it has already inspired our own performance-bug detection 
and performance failure diagnosis projects. 

Rule-based bug detection is widely used to detect functional bugs and security vulnerabilities. 
Inspired by our empirical study, we hypothesize that there are also statically 
checkable efficiency-related rules for performance bugs, 
violating which will lead to inefficient execution. 
These rules can be used to detect previously unknown performance bugs. 
To test our hypothesis, we manually examine fixed performance bugs, 
extract efficiency rules from performance bugs' patches, and implement static checkers to detect rules' violations. 
Our checkers find 332 previously unknown performance bugs. 
Some of found bugs have already been confirmed and fixed by developers. 
Our results demonstrate that rule-based performance-bug detection is a promising direction. 

Effectively diagnosing user-reported performance bugs is another key aspect of fighting performance bugs. 
Statistical debugging is one of the most effective failure diagnosis techniques designed for functional bugs. 
We explore the feasibility and design spaces to apply statistical debugging to performance failure diagnosis. 
We find that statistical debugging is a natural fit for diagnosing performance problems, 
which are often observed through comparison-based approaches and reported together 
with both good and bad inputs,
statistical debugging can effectively identify coarse-grained root causes 
for performance bugs under right types of design points, 
and special nature of performance bugs allows sampling to lower the 
overhead of runtime performance diagnosis without extending the diagnosis latency.

Performance bugs caused by inefficient loops contribute two thirds of user-reported performance bugs in our study. 
For them, coarse-grained root-cause information is not enough. 
To solve this problem, we first conduct an empirical study to understand what are fine-grained root
causes for inefficient loops in the real world. 
We then design LDoctor, which is a series of static-dynamic 
hybrid analysis routines that can help identify accurate fine-grained root-cause information. 
Sampling is leveraged to further lower diagnosis overhead, without hurting diagnosis accuracy or latency. 
Evaluation results show that LDoctor can cover most root-cause categories with good accuracy and small runtime overhead.

Our bug-detection technique and performance failure diagnosis techniques, 
guided by our empirical study, 
complement each other to significantly improve software performance. 

%\end{umiabstract}

\begin{nomenclature}
%\begin{description}

\begin{enumerate}

\item{{\textit{Performance bug:}}}
performance bugs are software implementation mistakes 
that can cause inefficient execution. 
We will use performance bugs
and performance problems interchangeably in this thesis 
following previous work in this 
area~\citep{Alabama,perf.fse10}.


\item{{\textit{Functional bug:}}}
functional bugs are software defects that lead to functional misbehavior, 
such as incorrect outputs, crashes, and hangs. 
Functional bugs include semantic bugs, 
memory bugs, concurrency bugs, and others~\citep{LiASID06}.



\item{{\textit{Efficiency rule:}}}
efficiency rules are principles inside programs, 
violations of which will lead to inefficient execution. 
Efficiency rules usually contain two components, 
which indicate where and how a particular code transformation 
can be conducted to improve performance, 
while preserving original functionality. 


\item{{\textit{Root-cause information:}}}
for a particular failure, root-cause information includes a static code region causing the failure, 
explanation why the failure happens, and potential fix suggestions.
We consider the output of fault localization as coarse-grained root causes, 
and consider failure explanation and fix suggestions as fine-grained root causes. 

\item{{\textit{Inefficient loop:}}}
inefficient loops are loops that conduct inefficient computation due to performance bugs.

\item{{\textit{Diagnosis latency:}}}
diagnosis latency is the time it takes to figure out the root cause of a failure.                    
It is measured by
how many failure runs are needed in order to conduct failure diagnosis. 


\item{{\textit{Runtime overhead:}}}
runtime overhead is measured by comparing normal execution with monitored execution. 


%\item{\makebox[0.75in][l]{$V$}}    Voltage

%\item{\makebox[0.75in][l]{\$}}     US Dollars

\end{enumerate}
%\end{description}
\end{nomenclature}

%\clearpage
\begin{abstract}
  Everyone wants software to run fast. 
Slow and inefficient software can easily frustrate end users and cause economic loss. 
Software-inefficiency problem has already caused several highly publicized failures. 
One major source of software's slowness is performance bug.
Performance bugs are software implementation mistakes that can cause inefficient execution. 
Performance bugs cannot be optimized away  by state-of-practice compilers. 
Many of them escape from in-house testing and manifest in front of end users, 
causing severe performance degradation and huge energy waste in the field. 
Performance bugs are becoming more critical, 
with the increasing complexity of modern software and workload,
the meager increases of single-core hardware performance, and the 
pressing energy concerns.
It is urgent to combat performance bugs.

This thesis works on three directions to fight performance bugs: 
performance bug understanding, rule-based performance-bug detection, 
and performance failure diagnosis. 

Building better tools needs a better understanding of performance bugs. 
In order to improve the understanding of performance bugs, 
we randomly sample 110 real-world performance bugs from 
5 large open-source software suites (Apache, Chrome, GCC, Mozilla and MySQL), 
and conduct the first empirical study on performance bugs. 
Our study is mainly performed to understand what are common root causes of performance bugs, 
how performance bugs are introduced, how to expose performance bugs and how to fix performance bugs. 
Important finds include: (1) there are dominating root causes and fix strategies for performance bugs, 
and root causes are highly correlated with fix strategies; 
(2) workload and API issues are two major reasons causing performance bugs to be introduced; 
(3) performance bugs require inputs with both special features and large scales to be exposed effectively. 
Our empirical study can guide future research on performance bugs, 
and it has already inspired our own performance-bug detection 
and performance failure diagnosis projects. 

Rule-based bug detection is widely used to detect functional bugs and security vulnerabilities. 
Inspired by our empirical study, we hypothesize that there are also statically 
checkable efficiency-related rules for performance bugs, 
violating which will lead to inefficient execution. 
These rules can be used to detect previously unknown performance bugs. 
To test our hypothesis, we manually examine fixed performance bugs, 
extract efficiency rules from performance bugs' patches, and implement static checkers to detect rules' violations. 
Our checkers find 332 previously unknown performance bugs. 
Some of found bugs have already been confirmed and fixed by developers. 
Our results demonstrate that rule-based performance-bug detection is a promising direction. 

Effectively diagnosing user-reported performance bugs is another key aspect of fighting performance bugs. 
Statistical debugging is one of the most effective failure diagnosis techniques designed for functional bugs. 
We explore the feasibility and design spaces to apply statistical debugging to performance failure diagnosis. 
We find that statistical debugging is a natural fit for diagnosing performance problems, 
which are often observed through comparison-based approaches and reported together 
with both good and bad inputs,
statistical debugging can effectively identify coarse-grained root causes 
for performance bugs under right types of design points, 
and special nature of performance bugs allows sampling to lower the 
overhead of runtime performance diagnosis without extending the diagnosis latency.

Performance bugs caused by inefficient loops contribute two thirds of user-reported performance bugs in our study. 
For them, coarse-grained root-cause information is not enough. 
To solve this problem, we first conduct an empirical study to understand what are fine-grained root
causes for inefficient loops in the real world. 
We then design LDoctor, which is a series of static-dynamic 
hybrid analysis routines that can help identify accurate fine-grained root-cause information. 
Sampling is leveraged to further lower diagnosis overhead, without hurting diagnosis accuracy or latency. 
Evaluation results show that LDoctor can cover most root-cause categories with good accuracy and small runtime overhead.

Our bug-detection technique and performance failure diagnosis techniques, 
guided by our empirical study, 
complement each other to significantly improve software performance. 

\end{abstract}

\clearpage\pagenumbering{arabic}

%%% END STUFF TAKEN FROM WITHESIS EXAMPLE FILE
