Writing efficient software is difficult.
Design and implementation defects can easily cause performance problems, 
leading to severe performance 
degradation in production runs. These problems annoy end users
and waste a huge amount of energy. Techniques that help
diagnose and fix performance problems are desired.

Unfortunately, existing performance diagnosis techniques are still preliminary,
failing to provide the desired diagnosis coverage, accuracy, and performance. 
Existing performance-bug detectors can accurately identify specific type
of inefficient computation, but are not ideal for performance diagnosis
in term of coverage and performance. 
Profiling tools cannot identify
the code regions that \textit{waste} the most resources, not to mention
explain why some resources are wasted, failing
both coverage and accuracy.
Recent work on statistical performance diagnosis can
identify coarse-grained root causes, such as which loop
is the root cause of a performance problem. However, it cannot provide 
fine-grained root cause information, such as why a loop is inefficient and
how developers might fix the problem.

In this paper, we first conduct an empirical study to understand what are
fine-grained performance problem root causes in real world. 
We then design a series of static-dynamic hybrid analysis routines that can
help identify accurate fine-grained performance root cause information.
We further use sampling techniques to lower our diagnosis overhead without
hurting diagnosis accuracy or latency. Evaluation using real-world performance
problems show that our tool can provide better coverage and accuracy than
existing performance diagnosis tools, with small run time overhead.
