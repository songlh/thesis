\section{Related Works}
\label{sec:related}

\subsection{Performance Diagnosis}
System developers always spend a lot of time in understanding software 
performance problems. Many tools and techniques have been built before to
help developers. 

Profilers are the most commonly used tools that help developers understand
performance.
Progresses are still been made in this area.
Some work helps associate complexity with bottleneck functions~\cite{Zaparanuks:2012:AP:2254064.2254074,Coppa:2012:IP:2254064.2254076};
some improves the profiling of calling context trees~\cite{D'Elia:2011:MHC:1993498.1993559}, 
some provides more precise profiling results~\cite{Mytkowicz:2010:EAJ:1806596.1806618}, and some targets new types of applications, 
like asynchronous programs~\cite{Ravindranath:2012:AMA:2387880.2387891} and 
interactive programs~\cite{Jovic:2011:CMY:2048066.2048081}.
Different from our work, profiling aims to tell where 
computation resources are spent, 
not where and why computation resources are wasted. 
The root cause code region of a performance problem often is not inside
the top-ranked function in the profiling result \cite{SongOOPSLA2014}.
Even if it is, developers still need to spend a lot of effort to understand
whether and what kind of computation inefficiency exists.

Recently, tools that go beyond profiling and provide more automated analysis
have been proposed.
IntroPerf~\cite{IntroPerf} helps automatically 
infer function call latencies based on
operating system tracers. StackMine~\cite{Han:2012:PDL:2337223.2337241} can
identify slow call-stack patterns inside event handlers. The work by
\citet{TaoAsplos2014} analyzes system traces to understand 
performance impact propagation and 
causality relationship among different system components. 
X-ray~\cite{Attariyan:2012:XAR:2387880.2387910} helps identify inputs
or configuration entries that are most responsible for performance problems.
Although these techniques are all very useful, 
they target different problems from \Tool. \Tool targets 
the inefficient loop problem, a common type of 
computation inefficiency problem that contributes to
about two thirds of real-world user-perceived performance problems studied
by previous work \cite{SongOOPSLA2014}, and tries to pin-point the exact
root cause and provide fix-strategy suggestions.

\Tool is most related to
the recent statistical performance debugging work
\citep{SongOOPSLA2014}, both trying to identify source-code level root causes
for user-perceived performance problems. 
This previous work can only identify which loop is most correlated with 
a performance symptom through statistical analysis, 
but cannot provide any information
regarding what type of inefficiency this loop contains.
\Tool complements it by
providing detailed root cause information and fix strategy suggestions through
program analysis. 

\subsection{Performance Bug Detection}

Bug detection finds previously unknown defects in programs that can
lead to unnecessary performance degradation. 

Many dynamic tools have been built to detect
run-time bloat~\cite{Dufour:2008:STC:1453101.1453111, Xu:2009:GFP:1542476.1542523, Xu:2010:DIC:1806596.1806616}, 
low-utility data structures~\cite{Xu:2010:FLD:1806596.1806617}, 
cacheable data~\cite{Cachetor}, performance slowdown caused by false sharing in
multi-thread programs~\cite{Liu:2011:SPD:2048066.2048070},
inefficient nested loops~\cite{Alabama},
input-dependent loops~\cite{xiao13:context}, and others. 
Static tools have also been proposed. Some static tools identify loops 
that are executing 
unnecessary iterations and hence could have been sped up by some extra
\texttt{break} statements
~\cite{CARAMEL,IsilDillig.PLDI15}. 
Some static tools \cite{PerfBug} identify performance defects that violate
specific efficiency rules, often related to API usage. 

As discussed in 
Section \ref{sec:intro}, these tools are all useful in improving software
performance, but are not suitable for performance diagnosis.
With different design goals from \Tool, these bug detection tools are not 
guided by any specific
performance symptoms. Consequently, they take different coverage-accuracy
trade-offs from \Tool. \Tool tries to cover a wide variety of root-cause 
categories and is more aggressive in identifying root-cause categories, because
it is only applied to a small number of loops that are known to be highly
correlated with the specific performance symptom under study.
Performance-bug detection tools has to be more conservative and tries hard
to lower false positive rates, because it needs to apply its analysis to the
whole program, instead of just a few loops. This requirement causes bug
detection tools to each focus on a specific root-cause category.
%In fact, most inefficient loops in our study are very difficult to be detected
%by a generic performance bug detectors --- it is very difficult to prove them to
%be inefficient without application-specific domain knowledge.
%For example, only a couple of inefficient loops out of the 45 studied
%in Table \ref{tab:root} can be detected by recent static inefficient
%loop checkers
%\cite{CARAMEL,IsilDillig.PLDI15}.
Performance bug detection tools also do not aim to provide fix suggestions.
For the few that do provide \cite{CARAMEL}, they only focus on very specific
fix patterns, such as adding a break into the loop. 
In addition, dynamic performance bug detectors do not try to achieve any
performance goals. None of them has tried applying sampling to their
bug detection and often leads to 10X slowdowns or more.

\subsection{Other Techniques to Fight Performance Bugs}

There are many test inputs generation techniques that help
performance testing. 
Wise~\cite{WISE} learns how to restrict conditional branches to get worst-case complexity from small inputs, 
and uses the learned policy to generate worst-case complexity inputs with 
larger size. 
EventBreak~\cite{EventBreak} generates performance testing inputs 
for web applications. 
SpeedGun~\cite{SpeedGun} generates performance regression tests for 
changed concurrent classes.

Some techniques aim to improve the test selection or prioritization during 
performance testing. 
Forepost~\cite{Forepost} is a test selection framework toward exposing more performance bottlenecks. 
It run applications under a small set of randomly selected test inputs to learn rules 
about which types of inputs are more likely to trigger intensive computation. 
Then it use learned rules to pick remaining test inputs in order to expose more performance bottlenecks. 
\citet{Huang:2014:PRT:2568225.2568232} study 100 performance regression issues. Based on the studying results, 
they design a performance risk analysis, which can help developers prioritize performance testing.  

All these techniques combat performance bugs from different aspects from performance diagnosis.
