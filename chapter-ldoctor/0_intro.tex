\section{Introduction}
\label{sec:6_intro}
\subsection{Motivation}

In Chapter~\ref{chap:sd}, 
we discussed how to apply statistical debugging to performance failure diagnosis. 
Given the symptom of 
a performance problem (e.g., two similar inputs leading to very
different execution speed),
statistical debugging, an approach widely used for diagnosing
functional failures~\citep{liblit03,liblit05,tarantula1}, can
accurately identify control-flow constructs that are most correlated with
the performance problem by comparing problematic runs and regular runs.
Unfortunately, for loop-related 
performance problems, which contribute to two thirds of user-reported
real-world performance problems studied in Chapter~\ref{chap:sd},
statistical debugging is not very effective.
Although it can identify the root-cause loop, it does not provide any 
information regarding why the loop is inefficient and hence is not very helpful
in fixing the performance problem.

Figure~\ref{fig:GCC27733} shows a real-world performance bug in GCC. 
Function \texttt{synth\_mult} is used to compute the best algorithm for 
multiplying \texttt{t}. It is so time consuming that developers tried 
speeding it up through memoization ---
hash-table
\texttt{alg\_hash} is used to remember which \texttt{t} has been processed
in the past and what is the processing result. 
Unfortunately, a small mistake in the type declaration of hash-table
entry \texttt{alg\_hash\_entry} causes memoization not to work when \texttt{t}
is larger than the maximum value type-\texttt{int} can represent.
As a result, under certain workload, \texttt{synth\_mult} conducts a lot of
redundant computation for same \texttt{t} values recursively, leading to
huge performance problems.  In practice, developers know the problem is inside
\texttt{synth\_mult} very early. However, since this function is 
complicated, it took them several weeks to figure out the root cause. 
If a tool can tell them not only which loop\footnote{Recursive functions are
handled similarly as loops in this chapter.}
is the root cause but also why the loop is inefficient (i.e., lot of redundant
computation and the need for better memoization in this example), 
the diagnosis and bug fixing
process would be much easier for developers. 

Clearly, more research is needed to improve the state of the art of performance
diagnosis. Diagnosis techniques that can provide fine-grained root-cause information for common performance problems,
especially loop-related performance problems, is well desired.

\subsection{Contributions}
This chapter presents a tool, \Tool, that can effectively provide fine-grained root causes for inefficient loops. 

\Tool targets the most common type of performance problems, 
inefficient loops, according to
empirical studies in Chapter~\ref{chap:study} and Chapter~\ref{chap:sd}.
\Tool also targets a challenging problem. Although
statistical debugging (Chapter~\ref{chap:sd})
can help identify suspicious loops that are highly correlated with 
user-perceived performance problems,
it provides no information about why the suspicious loop is inefficient,
not to mention providing fix suggestions. Although various
bug-detection techniques \citep{Cachetor,Alabama,CARAMEL} can help detect
loop-related performance bugs, they all suffer from generality problems,
covering only a specific type of loop problems each; the dynamic
tools \citep{Cachetor,Alabama} also suffer from performance
problem, imposing 10X slowdown or more.

\Tool can provide good coverage, accuracy and performance simultaneously:
1) Coverage. \Tool can cover most common root causes for inefficient loops. 
2) Accuracy. \Tool can accurately point out whether a loop is inefficient, explain why the loop is inefficient, 
and provide correct fix suggestions. 
3) Performance. \Tool incurs a low runtime overhead.

We build \Tool through the following three steps.

First, figuring out a taxonomy for the root causes of common inefficient loops.
Such a taxonomy is the prerequisite to developing a general and accurate
diagnosis tool. Guided by a thorough study of 45 real-world inefficient loop problems, we come up with a hierarchical root-cause taxonomy for
inefficient loops. 
Our empirical study shows that our taxonomy is general enough to cover
common inefficient loops, and also specific enough to help developers understand
and fix the performance problem. We will present more details 
in Section~\ref{sec:6_study}.

Second, building a tool(kit) \Tool that can automatically and accurately
identify the fine-grained root-cause type of a suspicious loop 
and provide fix-strategy suggestions. 
We achieve this following several principles:

\begin{itemize}
\item Focused checking. 
Different from performance-bug detection tools that blindly check the whole
software, \Tool focuses on performance symptoms and only check inefficient
loop candidates identified by existing tools, a statistical debugging
tool (Chapter~\ref{chap:sd}) in our current prototype. 
This focused study is crucial for \Tool to achieve high
accuracy.

\item Root-cause taxonomy guided design. To provide the needed coverage, we follow
the root-cause taxonomy discussed above and design analysis routines for 
every root-cause sub-category. Given a candidate inefficient loop, we will 
apply a series of analysis to it to see if it matches any type of inefficiency.

\item Static-dynamic hybrid analysis to balance performance and accuracy.
As we will see, static analysis alone cannot accurately identify 
inefficiency root causes, especially because some inefficiency problems only
happen under specific workload. However, pure dynamic analysis will 
cause too large 
runtime overhead. Therefore, we take a hybrid approach to achieve both
performance and accuracy goals.
\end{itemize}

Third, using sampling to further lower the runtime overhead of \Tool, without
degrading diagnosis capability. Random sampling is a natural fit for performance
diagnosis due to the repetitive nature of inefficient codes, especially
inefficient loops.

We have evaluated \Tool on 18 real-world performance problems. 
Evaluation results show that \Tool can accurately identify the detailed
root cause for all benchmarks and provide correct fix-strategy suggestion for
16 benchmarks. All these are achieved with low runtime overhead.

 
