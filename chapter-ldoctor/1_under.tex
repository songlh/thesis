\section{Root-Cause Taxonomy}
\label{sec:6_study}

The first task we are facing is to figure out a root-cause taxonomy for
real-world inefficient loops. 
Our taxonomy design is guided by checking inefficient loops 
in the benchmark suite discussed in Chapter~\ref{chap:study} and Chapter~\ref{chap:sd}. 
We also try to satisfy three requirements in our design.
(1) Coverage: covering
a big portion of real-world inefficient loop problems; 
(2) Actionability: each root-cause category should be informative enough 
to help developers
decide how to fix a performance problem; 
(3) Generality: too application-specific root causes will not work, as we
need to build diagnosis tools to automatically identify these root causes 
without developers' help.
In this section, we will first present our taxonomy, together with real-world
examples, followed by our empirical study about how a suite of real-world 
inefficient
loop problems fall into our root-cause categories.


\subsection{Taxonomy}
\label{sec:6_study_tax}
Our taxonomy contains two major root cause categories,
\textit{resultless} and \textit{redundancy}, and several sub-categories, 
as below.

\underline{\textit{\textbf{Resultless}}}
Resultless loops spend a lot of time in computation that does not 
produce any results that will be used after the loop.
Depending on when such resultless computation is conducted, we further
consider four sub-categories.

{\textbf{0*}}: 
This type of loops never produce any results in any iteration.
This type of loops should be rare in mature software systems.
They should simply be deleted from the program.
%Question: dynamic or static?

\begin{figure}
\codefig{Mozilla347306}
\caption{A resultless 0*1? bug in Mozilla}
\label{fig:Mozilla347306}
\end{figure}

{\textbf{0*1?}}:
This type of loops only produce results in the last iteration, if any. 
They are often related to search: they check a sequence of elements one
by one until the right one is found.
Not all loops of this type are inefficient. If they are, they are often
fixed by data-structure changes~\footnote{All bugs fixed by data-structure change 
are categorized as \emph{In-place Call Change} in Chapter~\ref{chap:study}.}.
Figure~\ref{fig:Mozilla347306} shows an example from Mozilla
JavaScript engine. 
The loop searches through an array containing source code information for the input \texttt{pc}. 
In practice, this loop needs to execute more than 10000 iterations for 
many long JavaScript encountered by users, which leads to huge 
performance problems. To fix this problem, 
developers simply replace array with hash table when processing long
JavaScript files.

\begin{figure}
\codefig{GCC46401}
\caption{A resultless [0$|$1]* bug in GCC}
\label{fig:GCC46401}
\end{figure}

{\textbf{[0$|$1]*}}:
Every iteration in this type of loops may or may not produce results.
Under certain workload, the majority of the loop iterations do not produce
any results and lead to performance problems perceived by end users.
The inefficiency problem caused by this type of loops can often be solved
by adding extra conditions to skip the loop under certain contexts.
Figure~\ref{fig:GCC46401} shows such an example from GCC.
%how users perceive this problem
%explain how every iteration may or may not produce any results
%explain the fix
Only when the if branch executed,
one iteration will generate results. 
The bug happens when GCC checks violations of sequence point rule. 
The user notices severe performance degradation when enabling the checking. 
The buggy loop is super-linear inefficient in terms of the number of operands
in one expression. The buggy input contains an expression, which is long and special. 
The loop takes a lot of iterations during bad run, but none of the iterations will 
generate results. 
The patch is to add extra checking before processing each operand. When it also has
the special feature like the bad input, the loop will be skipped. 

{\textbf{[1]*}}:
Loops in this category always generate results in almost all iterations. 
They are inefficient, because the results generated could be useless due to
some high-level semantic reasons.
Understanding and fixing this type of inefficiency problems often require
deep understanding of the program and are difficult to automate.
For example, several Mozilla performance problems are caused by 
loops that contain intensive GUI operations.
Although every iteration of these loops produce side effects, 
the performance problem forced developers to change the program 
and batch/skip some GUI operations.


\begin{figure}
\codefig{Mozilla477564}
\caption{A cross-loop redundant bug in Mozilla}
\label{fig:Mozilla477564}
\end{figure}


\underline{\textit{\textbf{Redundant}}}
Redundant loops spend a lot of time in repeating computation that is already
conducted.
Depending on the unit of the redundant computation, we further consider 
two sub-categories.

{\textbf{Cross-iteration Redundancy}}:
Loop iteration is the redundancy unit here:
one iteration repeats
repeating what was already done by an earlier iteration of the same loop.
Here, we consider recursive function calls as a loop, treating one function 
call instance as one loop iteration. 
This type of inefficiency is often fixed by memoization or batching, depending
on whether the redundancy involves I/O operations.
For example, GCC\#27733 shown in Figure~\ref{fig:GCC27733} 
is fixed by better memoization. 
%Another bug similar to GCC\#27733 is shown in Figure~\ref{fig:GCC1687}. 
%This bug is fixed by using a hash table to avoid visiting the same sub-tree twice.
Mozilla\#490742 in Figure~\ref{fig:Mozilla490742} represents a slightly different type of cross-iteration
redundancy. The inefficient loop in this case saves one URL into the ``Favorite
Links'' database in each iteration. One database transaction in each iteration
turns out to be to time consuming, with too much redundant work across
iterations. At the end, developers decide to batch all database updates into 
one big transaction, which speeds up some workload like 
bookmarking 50 tabs 
from popping up timeout windows to not blocking.
%xxx from several minutes
%to a couple of seconds.

{\textbf{Cross-loop Redundancy}}:
A whole loop is the redundancy unit here:
one dynamic instance of a loop spends a big chunk, if not all, of its
computation in repeating the work already done by an
earlier instance of the same loop.
Developers often fix this type of inefficiency problems through memoization:
caching the earlier computation results and skip following redundant loops.
Mozilla\#477564 shown in Figure~\ref{fig:Mozilla477564} is an example for this type of bugs. 
The buggy loop is to count how many previous siblings of the input \texttt{aNode} have the same name and URI. 
There is an outer loop for the buggy one, and the outer loop will update \texttt{aNode} by using its next siblings. 
The bug is fixed by saving the calculated count for each node. 
A new count value is calculated by adding one to the saved count value of the nearest previous sibling with the same name and URI.


\subsection{Empirical study}
\label{sec:6_tax_study}
Having presented our taxonomy above, we will see how it works for a set of
real-world inefficient loop problems collected in Chapter~\ref{chap:study} and Chapter~\ref{chap:sd}. 
Specifically, we want to check the 
coverage and the actionability of our taxonomy presented above:
(1) How common are real-world performance problems caused by the 
patterns discussed above?
(2) How are real-world problems fixed by developers? Can we predict their
fix strategies based on the root-cause pattern?

\subsubsection{Methodology}
\begin{table}[h!]
\scriptsize
\centering
\begin{tabular}{@{\hspace{3pt}}l@{\hspace{3pt}}@{\hspace{3pt}}c@{\hspace{3pt}}}
\toprule
Application Suite Description (language) & \# Bugs \\
\midrule                            
{\bf Apache Suite} 	 & 11\\
%\cline{1-1}
{HTTPD:	Web Server (C)	}& \\
{TomCat:  Web Application Server (Java)}& \\
{Ant:	Build management utility (Java)}& \\
%\hline
%JMeter	& Load test utility (Java) & \\
\midrule                            
{\bf Chromium Suite} Google Chrome browser (C/C++) & 4\\
\midrule
%\multicolumn{2}{|l|}
{\bf GCC Suite}  GCC \& G++ Compiler (C/C++)     & 8\\
\midrule
{\bf Mozilla Suite}  & 12\\
%\cline{1-1}
{Firefox: Web Browser (C++, JavaScript)}& 	\\
{Thunderbird: Email Client (C++, JavaScript)}& \\
\midrule
{\bf MySQL Suite}     & 10	\\
%\cline{1-1}
{Server: Database Server (C/C++)}&  	\\
%\cline{1}
{Connector: DB Client Libraries (C/C++/Java/.Net)} &  	\\
\midrule
{\bf Total}	   & 45 \\
\bottomrule
\end{tabular}
\caption{Applications and bugs used in the study}
\label{tab:6_app_bug}
\end{table}

In Chapter~\ref{chap:study}, we studied the on-line bug
databases of five representative open-source software projects, as 
shown in Table~\ref{tab:6_app_bug}. 
Through a mix of random sampling and 
manual inspection, we
found 65 performance problems that are perceived and reported by users in Chapter~\ref{chap:sd}. 
Among these 65 problems, 45 problems are related to inefficient loops and 
hence are the target of the study 
here\footnote{The definition of ``loop-related'' in this chapter is a little
bit broader than in Chapter~\ref{chap:sd}, which only considers
43 problems as loop-related. }.

\subsubsection{Observations}
\label{sec:6_study_ob}
\begin{table*}[tb!]
\begin{adjustwidth}{-.5in}{-.5in}
\scriptsize
\centering
{
\begin{tabular}{lccccccl}
\toprule
&Apache&Chrome&GCC&Mozilla&MySQL&Total&Fix Strategy\\
\midrule
Total \# of loop-related bugs & 11 & 4 & 8 & 12 & 10 & 45 &  \\
\midrule
\multicolumn{8}{c}{\# of {\textit {Resultless}} bugs}\\
\multicolumn{1}{l}{ {\bf 0*} }
&0&0&0&0&0&0&\\
\multicolumn{1}{l}{ {\bf 0*1?} }
&0&0&0&2&3&5&C(4)$|$S(1)\\
\multicolumn{1}{l}{{\bf [0$|$1]*}}
&0&1&1&1&1&4&S(4)\\
\multicolumn{1}{l}{{\bf 1*}}
&1&2&3&6&3&15& B(4)$|$S(4)$|$O(7)\\
%&0&2&0&5&0&7&B(4)$|$S(3)\\
%&1&0&3&1&3&8&\\
\midrule
\multicolumn{8}{c}{ \# of {\textit {Redundant}} bugs}\\
\multicolumn{1}{l}{Cross-{\bf iteration} redundancy}
&7&1&2&1&1&12&B(4)$|$M(8)\\
\multicolumn{1}{l}{ Cross-{\bf loop} redundancy}
&3&0&2&2&2&9&B(4)$|$M(5)\\
  %MySQL15811 is moved from 1* to here; consider it as fixed by M
%\hline
%\multicolumn{8}{|c|}{ \# of {\textit {Other}} bugs}\\
%\multicolumn{1}{|l}{Not in above categories}
\bottomrule
\end{tabular}
}
\end{adjustwidth}
\caption{Number of bugs in each root-cause category. 
B, M, S, C, and O represent different fix strategies:
B(atching),  
M(emoization), 
S(kipping the loop),
C(hange the data structure), and O(thers). The numbers in the 
parentheses denote the number of problems that are fixed using specific
fix strategies.}
\label{tab:6_root}
\end{table*}



\underline{Are the root-causes in our taxonomy common?}
The answer is \textit{yes}.
Resultless loops are about as common as redundant loops
(24 vs. 21).
Intuitively, 0* loops, where no iteration produces any results,
are rare in mature software. In fact, no bugs in this
benchmark suite belong to this category.
All other root-cause sub-categories are well represented.


\underline{How are real-world performance bugs fixed?}
Overall, the fix strategies are well aligned with root-cause patterns.
For example, 
all inefficient loops with resultless [0$|$1]* are
fixed by skipping the loop under certain contexts~\footnote{All bugs fixed by skipping the loop are 
categorized as \emph{Change Condition} in Chapter~\ref{chap:study}.},
and all inefficient loops with redundant root causes are fixed either by 
memoization or batching the computation. 
In fact, as we will discuss later, simple analysis can tell whether 
memoization or batching should be used to fix a problem.
The only problem is that there are no silver bullets for 1* loops.
%all inefficient loops with xx root cause are fixed by xxx.
%xxx

\paragraph{Implications}
As we can see, resultless loops and redundant loops are both common reasons
that cause user-perceived performance problems in practice. The root-cause
categories discussed above also match with developers' fix strategy well ---
with a little bit amount of extra analysis, one can pretty much predict the fix strategy based on the root-cause pattern.
Consequently, tools that cover these root causes will have high
chances to satisfy the coverage and accuracy requirements of performance 
diagnosis.

%TODO even for the other, there are things we could do ...

\subsubsection{Caveats} 
Just as previous empirical study work, 
our empirical study above needs to be interpreted with our methodology in mind.
As we discussed in Chapter~\ref{chap:study} and Chapter~\ref{chap:sd}, 
we use developers tagging and
on-line developer/user discussion to judge whether a bug report is about
performance problems and whether 
the performance problem under discussion is noticed and reported by users
or not.
We follow the methodology used in Chapter~\ref{chap:sd} to judge 
whether the root-cause of a performance problem is related to loops or not.
We do not intentionally ignore any aspect of loop-related performance problems. 
Some loop-related performance problems may never be noticed by end users
or never be fixed by developers, and hence skipped by our study. However,
there are no conceivable ways to study them. 

We believe that the bugs in our study provide a representative sample of the 
well-documented and
fixed performance bugs that are user-perceived and loop-related in the studied 
applications. 
Since we did not set up the root-cause taxonomy to fit particular
bugs in this bug benchmark suite, we believe our taxonomy and diagnosis
framework presented below will go beyond these sampled performance bugs. 

