\section{Performance-bug Detection}
\label{sec:2_detect}

Many performance-bug detection tools have been proposed recently.
They each aims to find a specific type of hidden
performance bugs before the bugs lead to performance problems observed by
end users.

Some tools~\citep{BloatFSE2008, XuBloatPLDI2009, XuBloatPLDI2010, PerfBlower}
detect runtime bloat, a common performance problem in object-oriented 
applications.
Some tools~\citep{LeakChaser, JavaMemoryLeak} detect unnecessary references leading to memory leak. 
\citet{XuDataStructure} target 
low-utility data structures with unbalanced costs and benefits.
Cachetor~\citep{Cachetor} detects cacheable computation through dynamic dependence profiling and value profiling. 
A series abstracts are proposed to reduce the detection overhead. 
\citet{Reusable} detects reuseable object, 
data structures or data contained in the data structures. 
In a follow up work~\citep{Resurrector}, 
a new object lifetime profiling technique is proposed to improve the reusability detection results. 
\citet{ORMPatterns} detect database-related performance anti-patterns,
like fetching excessive data from database and issuing queries that could have
been aggregated.
WAIT~\citep{WAIT} focuses on bugs that
block the application from making progress.
\citet{falsesharing} build two tools to attack the false 
sharing problem in multi-threaded software.

Similar to our detection work in Chapter~\ref{chap:detec}, 
\citet{SmartphoneStudy} extract two rules from studied performance bugs, 
and detect violations to report unknown performance bugs. 
The bug patterns they can detect are lengthy operations 
in main thread and frequently-invoked heavy-weight callback. 
Different from our work, their bug detection technique focuses on 
performance bugs inside Android applications. 

There are techniques targeting inefficient loops.
Toddler~\citep{Alabama} detects inefficient nested loops by monitoring memory read instructions. 
Given a nested loop, read value sequences from the same instructions inside the inner 
loop are compared across different iterations of the outer loop, 
and the nested loops will be reported as inefficiency, when repetitive value sequences are found.
Caramel~\citep{CARAMEL} presents 4 static checkers to detect performance bugs with Cond-Break fixes. 
Every detected bug is associated with a loop and a condition. 
When the condition becomes true, the loop will not generate any results visible after the loop terminates. 
Therefore, the loop can be sped up by adding an extra break statement.
~\citet{xiao13:context} profile programs to calculate complexity for each basic block, 
and identify a loop as input-dependent loop, if the loop can cause complexity change between outside the loop and inside the loop. 
CLARITY~\citep{CLARITY} can statically detect asymptotic 
performance bugs caused by traversing a data structure repeatedly. 

These tools are all useful in improving software performance, 
but they target different types of performance bugs 
from the performance-bug detection technique (Chapter~\ref{chap:detec}) in this thesis. 
With different design goals, these performance-bug detection tools are not guided by any specific
performance symptoms. Consequently, they take different
coverage-accuracy trade-offs from performance diagnosis techniques (Chapter~\ref{chap:ldoctor}).
Performance diagnosis techniques try
to cover a wide variety of root-cause categories and are more
aggressive in identifying root-cause categories, because they are
only applied to a small number of places that are known to
be highly correlated with the specific performance symptom
under study. Performance-bug detection tools have to be more
conservative and try hard to lower false positive rates, 
because they need to apply their analysis to the whole program. 

